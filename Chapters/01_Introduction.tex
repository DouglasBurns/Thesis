\newpage\null\thispagestyle{empty}
\newpage\null\thispagestyle{empty}

\chapter{Introduction}
\label{ch:Introduction}
\pagenumbering{arabic}

For hundreds of years, the people of this world have tried to answer the most fundamental questions we can ask.
What is consciousness?
Why do we dream?
Are we alone in the Universe?
How did life begin?
How did the Universe begin?
What is the Universe made of and how does it work?
These are not easy questions.
In truth, we can not definitively answer any of them.
This does not, or ever will, dampen our spirits in the quest to answer them.

The standard model of particle physics is our best attempt to describe our Universe on the smallest scale and how it evolves.
It is an extremely successful, well tested model that describes the existence and behaviours of all the observed particles which form the constituents of the visible Universe.
Indeed, it predicts accurately how particles interact at energy scales over many orders of magnitude; from simple magnetic compasses and powerful lasers to the high energy collisions experienced at current particle colliders.

It is, nevertheless, incomplete.
It fails to include the simple action of an apple dropping in a gravitational field, it fails to describe more than the baryonic matter in the universe, a total of approximately 4\% and it fails to describe why we live in such a matter dominated Universe.
But these failures are seen as opportunities by physicists, searching for new particles and new interactions that can be used to describe Nature.

Perhaps one of the most exciting areas of the the standard model to look at in detail is that orientated around the top quark.
The top quark is the most massive particle of the standard model and will decay without forming a hadron, and as such it allows us to study its bare quark properties.
Its large mass also means that the top quark is sensitive to new physics, either by direct production or from interference produced from particles produced at higher energy scales.
Aside from searching for new physics, the top quark appears in, and can be a major background to, many rare standard model processes.
Observing these processes will further validate the standard model. 
Many of the possible new physics processes will have a direct effect on the top quark pair production cross section.

For these reasons and more, it is vital to know the production cross section to the highest degree of precision possible, not only for the inclusive cross section, but for differential cross sections too.
The breakdown of cross sections as a function of a particular variable provides valuable information on how well the top quark pair system is modelled.

This thesis sets out to measure the differential cross sections of top quark pair production with respect to several kinematic event variables.
It does so in the single lepton decay channel of the top quark and presents the measurements in a phase space similar to that accessible by the Compact Muon Solenoid detector and with respect to detectable particles.

In Ch.~\ref{ch:Theory}, a brief summary of the standard model, its shortcomings and top quark physics is discussed. 
Chapter~\ref{ch:MC} discusses the modelling of interactions and lists the simulated datasets used in this thesis.
From simulated to real interactions, Ch.~\ref{ch:LHCCMS} discusses the Large Hadron Collider, the Compact Muon Solenoid (\CMS) experiment and the data collection performed.
Once the raw data is collected, it must be reconstructed into usable analysis objects.
This is discussed in Ch.~\ref{ch:ObjectReconstruction}.
Once the analysis objects have been created, an analysis can be formed. 
Chapter~\ref{ch:analysis} discusses the event selection performed, the variable definitions and the corrections which need to be applied to simulated and real events.
The yield of top quark pair events reconstructed for each variable is calculated.
Chapter~\ref{ch:unfolding} describes the unfolding procedure used to remove the detector response imprinted on the data and to present the results to particle level in the visible phase space.
The unfolded yields are used in the calculation of normalised and absolute cross sections.
The uncertainties, both statistical and systematic, are discussed in Ch.~\ref{ch:uncertainty}.
The final results are presented in Ch.~\ref{ch:xsection} along with goodness-of-fit tests performed between the unfolded data and our top quark pair production models taking into account the correlations of uncertainties between bins of the measurement.
A study on the effect of regularisation on the unfolding is shown.
Finally, in Ch.~\ref{ch:outlook} the future prospects beyond this thesis are discussed.

It is worth noting that natural units are used in this thesis where $c = \hbar = 1$. This means that mass, momentum and energy are all measured using\unit{eV}.






% Always remember, if in doubt the answer is 42.



