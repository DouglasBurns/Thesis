\section{Introduction TODO Change title}
\label{sec:analysisIntro}

With approximately 30 million \tt{} pairs produced in 35.9\fbinv{} of data, collected by CMS during 2016, detailed studies of the production of \tt{} events can be performed. In this thesis, measurements of the normalised and absolute differential \tt{} production cross section, with respect to global event distributions, in the single lepton final state are presented. The data collected was taken at \com{} = 13\TeV{} with a bunch spacing of 25\ns{}. The events selected in this analysis contain one isolated, high \pt{}, electron or muon in the final state, with four jets of which at least two are tagged as coming from a b quark.

The measurements are presented at particle level in a visible phase space, which are described in Section TODO

As \tt{} production is a significant background in many searches for physics beyond the standard model TODO(defined SM?), it is vital that it is well understood. These measurements are important in the validation the current theoretical models of \tt{} production and decay, that are implemented in the current state-of-the-art MC generators. 
% With sufficient understanding, rare standard model processes can also be measured such as \tt{} production in association with a W, Z or H boson.

This analysis was published in TODO TODO(Ref this Paper). TODO(Add ref for older paper?legacy EA.) It is complimentary to other analyses TODO(refs. rochester, dilepton etc...)

\subsection{Global Event Distributions}
\label{ssec:gEventDist}

Global event distributions are distributions which do not require the reconstruction of the \tt{} system, instead considering the event in its entirety. This analysis measures the \LETA{} and \LPT{} of the isolated lepton, the scalar sum of \JPT{} above 30\GeV{} (\HT{}), the magnitude of the missing transverse momentum (\MET{}), the magnitude of total activity in the event (\ST{}=\HT{}+\MET{}+\LPT{}), the magnitude of the \pt{} of the leptonically decaying \Wboson{} boson (\WPT{}) and the jet multiplicity above 30\Gev{} (\NJET{}).

TODO(INSERT NJETTINESS)


\subsection{Simulation of the Signal and Background}
\label{ssec:simSigBkg}

The simulation of \tt{} production is generated using the \Powheg{} NLO matrix-element generator, combined with \Pythia{} parton showerer, using the \CUET{} tune. The production of \tt{} is also simulated, by combining the output of \Powheg{} with the \Herwigpp{} parton showerer, using the EE5C tune. Two additional samples of \tt{} production were produced using the \mgamc{} matrix-element generator. The first is generated at LO accuracy with up to three additional partons and combined with \Pythia{} using the \CUET{} tune, using the MLM matching algorithm. The second is generated at NLO accuracy with up to two additional partons and again combined with \Pythia{} using the \CUET{} tune, using the FXFX matching algorithm

These generators are referred to as \PowhegPythia{}, \PowhegHerwig{}, \mgamcLO{} and \mgamcNLO{} respectively.


