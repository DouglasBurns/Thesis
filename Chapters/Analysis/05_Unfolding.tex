\section{Unfolding}
\label{sec:unfold}

What is the process of unfolding

\subsection{TUnfold}
\label{ssec:TUnfold}
% Regularisation

\subsection{Binning and Response Matrices}
\label{ssec:bin}

\subsection{Best Regularisation Parameter}
\label{ssec:bestTau}

\subsection{Unfolding Tests}
\label{ssec:uTests}

Unfolding tests are a vital way to ensure that the unfolding method does not bias or mistreat the uncertainties on the raw data passed to it. The tests are performed using the MC-truth information and are explained in more detail in the following Sections. TODO(Capital S?). 


\subsubsection{Pull Tests}
\label{sssec:pulls}
	% Measured distributions from one varied response (X and Y projections)
	% Measured was unfolded by second varied response
	% Pull in bin taken as |unfolded-truth|/sigma_unfolded.

	Pull distributions were created from unfolding TODO N pseudo experiments. For each pseudo experiment the measured and truth distributions were generated by taking the profile from a response matrix that has been varied. The measured distributions were then unfolded using another varied response matrix (TODO response matrices related to much?) and the pull measured, shown in Eq. \ref{eq:pull}, for each bin $i$. If the resulting mean and width of the pull distribution is close to zero and one respectively, then there is confirmation that the statistical uncertainties have been treated correctly by the response matrices produced using the TUnfold method. TODO move explanation of why first

	\begin{equation}
	\label{eq:pull}
	\text{Pull}^{i}=\frac{|X^{i}_{\text{truth}}-X^{i}_{\text{unf}}|}{\sigma^{i}_{\text{unf}}}
	\end{equation}

	An example of the pull distribution for TODO(Variable) in the TODO(channel) is shown in Fig. \ref{fig:pullExample}. TODO(CHANGE PICTURE). As shown, the mean is $\approx0$ and the width $\approx1$, meaning TODO(). Figures \ref{fig:Pullse} and \ref{fig:Pullsmu}, show the mean and width of the pull distributions in each bin of each global event variable.
	
	\begin{figure*}[hp]
	\centering
	\includegraphics[width=0.32\textwidth]{/Users/db0268/Mount/SoolinScratch/DPS/DPSTestingGround/DailyPythonScripts/data_thesis/plots/unfolding/pulls/NJets_electron.pdf} 
	\includegraphics[width=0.32\textwidth]{/Users/db0268/Mount/SoolinScratch/DPS/DPSTestingGround/DailyPythonScripts/data_thesis/plots/unfolding/pulls/HT_electron.pdf} 
	\includegraphics[width=0.32\textwidth]{/Users/db0268/Mount/SoolinScratch/DPS/DPSTestingGround/DailyPythonScripts/data_thesis/plots/unfolding/pulls/ST_electron.pdf} \\
	\includegraphics[width=0.32\textwidth]{/Users/db0268/Mount/SoolinScratch/DPS/DPSTestingGround/DailyPythonScripts/data_thesis/plots/unfolding/pulls/MET_electron.pdf} 
	\includegraphics[width=0.32\textwidth]{/Users/db0268/Mount/SoolinScratch/DPS/DPSTestingGround/DailyPythonScripts/data_thesis/plots/unfolding/pulls/WPT_electron.pdf} \\
	\includegraphics[width=0.32\textwidth]{/Users/db0268/Mount/SoolinScratch/DPS/DPSTestingGround/DailyPythonScripts/data_thesis/plots/unfolding/pulls/lepton_pt_electron.pdf} 
	\includegraphics[width=0.32\textwidth]{/Users/db0268/Mount/SoolinScratch/DPS/DPSTestingGround/DailyPythonScripts/data_thesis/plots/unfolding/pulls/abs_lepton_eta_coarse_electron.pdf} 
	\caption[The pull mean and widths in relation to the bin numbers of the event variables in the \eJets{} channel. The 5000 pseudo experiments are generated from the \powhegpythia{} response matrix.]{The pull mean and widths in relation to the bin numbers of the event variables in the \eJets{} channel. The 5000 pseudo experiments are generated from the \powhegpythia{} response matrix.}
	\label{fig:Pullse}
\end{figure*}

\begin{figure*}[hp]
	\centering
	\includegraphics[width=0.32\textwidth]{/Users/db0268/Mount/SoolinScratch/DPS/DPSTestingGround/DailyPythonScripts/data_thesis/plots/unfolding/pulls/NJets_muon.pdf} 
	\includegraphics[width=0.32\textwidth]{/Users/db0268/Mount/SoolinScratch/DPS/DPSTestingGround/DailyPythonScripts/data_thesis/plots/unfolding/pulls/HT_muon.pdf} 
	\includegraphics[width=0.32\textwidth]{/Users/db0268/Mount/SoolinScratch/DPS/DPSTestingGround/DailyPythonScripts/data_thesis/plots/unfolding/pulls/ST_muon.pdf} \\
	\includegraphics[width=0.32\textwidth]{/Users/db0268/Mount/SoolinScratch/DPS/DPSTestingGround/DailyPythonScripts/data_thesis/plots/unfolding/pulls/MET_muon.pdf} 
	\includegraphics[width=0.32\textwidth]{/Users/db0268/Mount/SoolinScratch/DPS/DPSTestingGround/DailyPythonScripts/data_thesis/plots/unfolding/pulls/WPT_muon.pdf} \\
	\includegraphics[width=0.32\textwidth]{/Users/db0268/Mount/SoolinScratch/DPS/DPSTestingGround/DailyPythonScripts/data_thesis/plots/unfolding/pulls/lepton_pt_muon.pdf} 
	\includegraphics[width=0.32\textwidth]{/Users/db0268/Mount/SoolinScratch/DPS/DPSTestingGround/DailyPythonScripts/data_thesis/plots/unfolding/pulls/abs_lepton_eta_coarse_muon.pdf} 
	\caption[The pull mean and widths in relation to the bin numbers of the event variables in the \muJets{} channel. The 5000 pseudo experiments are generated from the \powhegpythia{} response matrix.]{The pull mean and widths in relation to the bin numbers of the event variables in the \muJets{} channel. The 5000 pseudo experiments are generated from the \powhegpythia{} response matrix.}
	\label{fig:Pullsmu}
\end{figure*}


\subsubsection{Closure and Bias Tests}
\label{sssec:bias}
	TODO:MORE INFO ON WHAT EXACTLY BIAS TEST IS AND DOES
	% 
	The response matrix, as it is constructed directly from a \tt{} MC model, is model dependent. This means that a model that poorly describes the data can introduce a bias into the unfolded distributions. To test the size of any bias that may be introduced, the top \pt{} spectrum in the \PowhegPythia{} MC is reweighted up and down to cover any discrepency between the \PowhegPythia{} MC and the data, according to Eq. \ref{eq:reweight}.

	\begin{equation}
	\label{eq:reweight}
	w(t/\overline{t})=1+(p_{T}^{t/\overline{t}} \pm 100) \times 0.001
	\end{equation}

	The reweighted MC is then unfolded using the central response matrix, generated from the \PowhegPythia{} sample, before being compared to the MC truth. The bias is calculated by taking the ratio of the unfolded reweighted samples to the respective truth information. Figures \ref{fig:Reweightingse} and \ref{fig:Reweightingsmu} show the reweighting of the central \PowhegPythia{} MC compared to the unfolded data. TODO ADD MORE STUFF HERE?. Figures \ref{fig:ClosureBiase1}, \ref{fig:ClosureBiase2}, \ref{fig:ClosureBiasmu1} and \ref{fig:ClosureBiasmu2} show the closure for unfolding the reweighted samples and the additional alternate \tt{} generators under study, with the corresponding bias introduced.

	\begin{figure*}[htpb]
	\centering
	\includegraphics[width=0.32\textwidth]{/Users/db0268/Mount/SoolinScratch/DPS/DPSTestingGround/DailyPythonScripts/plots/unfolding/reweighting_check/Reweighting_check_electron_HT_at_13TeV.pdf}
	\includegraphics[width=0.32\textwidth]{/Users/db0268/Mount/SoolinScratch/DPS/DPSTestingGround/DailyPythonScripts/plots/unfolding/reweighting_check/Reweighting_check_electron_ST_at_13TeV.pdf}
	\includegraphics[width=0.32\textwidth]{/Users/db0268/Mount/SoolinScratch/DPS/DPSTestingGround/DailyPythonScripts/plots/unfolding/reweighting_check/Reweighting_check_electron_MET_at_13TeV.pdf} \\
	\includegraphics[width=0.32\textwidth]{/Users/db0268/Mount/SoolinScratch/DPS/DPSTestingGround/DailyPythonScripts/plots/unfolding/reweighting_check/Reweighting_check_electron_WPT_at_13TeV.pdf}
	\includegraphics[width=0.32\textwidth]{/Users/db0268/Mount/SoolinScratch/DPS/DPSTestingGround/DailyPythonScripts/plots/unfolding/reweighting_check/Reweighting_check_electron_lepton_pt_at_13TeV.pdf} \\
	\includegraphics[width=0.32\textwidth]{/Users/db0268/Mount/SoolinScratch/DPS/DPSTestingGround/DailyPythonScripts/plots/unfolding/reweighting_check/Reweighting_check_electron_abs_lepton_eta_coarse_at_13TeV.pdf} 
	\includegraphics[width=0.32\textwidth]{/Users/db0268/Mount/SoolinScratch/DPS/DPSTestingGround/DailyPythonScripts/plots/unfolding/reweighting_check/Reweighting_check_electron_NJets_at_13TeV.pdf}
	\caption[Reweighting of the \PowhegPythia{} MC with respect to the unfolded data for \HT{}, \ST{}, \MET{} (top), \WPT{}, \LPT{} (middle), \LETA{} and \NJET{} (bottom) in the electron channel.]{Reweighting of the \PowhegPythia{} MC with respect to the unfolded data for \HT{}, \ST{}, \MET{} (top), \WPT{}, \LPT{} (middle), \LETA{} and \NJET{} (bottom) in the electron channel.}
	\label{fig:Reweightingse}
\end{figure*}

\begin{figure*}[htpb]
	\centering
	\includegraphics[width=0.32\textwidth]{/Users/db0268/Mount/SoolinScratch/DPS/DPSTestingGround/DailyPythonScripts/plots/unfolding/reweighting_check/Reweighting_check_muon_HT_at_13TeV.pdf} 
	\includegraphics[width=0.32\textwidth]{/Users/db0268/Mount/SoolinScratch/DPS/DPSTestingGround/DailyPythonScripts/plots/unfolding/reweighting_check/Reweighting_check_muon_ST_at_13TeV.pdf} 
	\includegraphics[width=0.32\textwidth]{/Users/db0268/Mount/SoolinScratch/DPS/DPSTestingGround/DailyPythonScripts/plots/unfolding/reweighting_check/Reweighting_check_muon_MET_at_13TeV.pdf} \\
	\includegraphics[width=0.32\textwidth]{/Users/db0268/Mount/SoolinScratch/DPS/DPSTestingGround/DailyPythonScripts/plots/unfolding/reweighting_check/Reweighting_check_muon_WPT_at_13TeV.pdf} 
	\includegraphics[width=0.32\textwidth]{/Users/db0268/Mount/SoolinScratch/DPS/DPSTestingGround/DailyPythonScripts/plots/unfolding/reweighting_check/Reweighting_check_muon_lepton_pt_at_13TeV.pdf} \\
	\includegraphics[width=0.32\textwidth]{/Users/db0268/Mount/SoolinScratch/DPS/DPSTestingGround/DailyPythonScripts/plots/unfolding/reweighting_check/Reweighting_check_muon_abs_lepton_eta_coarse_at_13TeV.pdf}
	\includegraphics[width=0.32\textwidth]{/Users/db0268/Mount/SoolinScratch/DPS/DPSTestingGround/DailyPythonScripts/plots/unfolding/reweighting_check/Reweighting_check_muon_NJets_at_13TeV.pdf}
	\caption[Reweighting of the \PowhegPythia{} MC with respect to the unfolded data for \HT{}, \ST{}, \MET{} (top), \WPT{}, \LPT{} (middle), \LETA{} and \NJET{} (bottom) in the muon channel.]{Reweighting of the \PowhegPythia{} MC with respect to the unfolded data for \HT{}, \ST{}, \MET{} (top), \WPT{}, \LPT{} (middle), \LETA{} and \NJET{} (bottom) in the muon channel.}
	\label{fig:Reweightingsmu}
\end{figure*}
	\begin{figure*}[htpb]
	\centering
	\includegraphics[width=0.32\textwidth]{/Users/db0268/Mount/SoolinScratch/DPS/DPSTestingGround/DailyPythonScripts/plots/unfolding/closure_test/TUnfold/number_of_unfolded_events_electron_closure_test_for_HT_at_13TeV.pdf}
	\includegraphics[width=0.32\textwidth]{/Users/db0268/Mount/SoolinScratch/DPS/DPSTestingGround/DailyPythonScripts/plots/unfolding/bias_test/Bias_normalised_xsection_electron_HT_at_13TeV.pdf} \\
	\includegraphics[width=0.32\textwidth]{/Users/db0268/Mount/SoolinScratch/DPS/DPSTestingGround/DailyPythonScripts/plots/unfolding/closure_test/TUnfold/number_of_unfolded_events_electron_closure_test_for_ST_at_13TeV.pdf}
	\includegraphics[width=0.32\textwidth]{/Users/db0268/Mount/SoolinScratch/DPS/DPSTestingGround/DailyPythonScripts/plots/unfolding/bias_test/Bias_normalised_xsection_electron_ST_at_13TeV.pdf} \\
	\includegraphics[width=0.32\textwidth]{/Users/db0268/Mount/SoolinScratch/DPS/DPSTestingGround/DailyPythonScripts/plots/unfolding/closure_test/TUnfold/number_of_unfolded_events_electron_closure_test_for_MET_at_13TeV.pdf}
	\includegraphics[width=0.32\textwidth]{/Users/db0268/Mount/SoolinScratch/DPS/DPSTestingGround/DailyPythonScripts/plots/unfolding/bias_test/Bias_normalised_xsection_electron_MET_at_13TeV.pdf} \\
	\includegraphics[width=0.32\textwidth]{/Users/db0268/Mount/SoolinScratch/DPS/DPSTestingGround/DailyPythonScripts/plots/unfolding/closure_test/TUnfold/number_of_unfolded_events_electron_closure_test_for_WPT_at_13TeV.pdf}
	\includegraphics[width=0.32\textwidth]{/Users/db0268/Mount/SoolinScratch/DPS/DPSTestingGround/DailyPythonScripts/plots/unfolding/bias_test/Bias_normalised_xsection_electron_WPT_at_13TeV.pdf} \\
	\caption[help]{help}
	\label{fig:ClosureBiase1}
\end{figure*}

\begin{figure*}[htpb]
	\centering
	\includegraphics[width=0.32\textwidth]{/Users/db0268/Mount/SoolinScratch/DPS/DPSTestingGround/DailyPythonScripts/plots/unfolding/closure_test/TUnfold/number_of_unfolded_events_electron_closure_test_for_lepton_pt_at_13TeV.pdf}
	\includegraphics[width=0.32\textwidth]{/Users/db0268/Mount/SoolinScratch/DPS/DPSTestingGround/DailyPythonScripts/plots/unfolding/bias_test/Bias_normalised_xsection_electron_lepton_pt_at_13TeV.pdf} \\
	\includegraphics[width=0.32\textwidth]{/Users/db0268/Mount/SoolinScratch/DPS/DPSTestingGround/DailyPythonScripts/plots/unfolding/closure_test/TUnfold/number_of_unfolded_events_electron_closure_test_for_abs_lepton_eta_coarse_at_13TeV.pdf}
	\includegraphics[width=0.32\textwidth]{/Users/db0268/Mount/SoolinScratch/DPS/DPSTestingGround/DailyPythonScripts/plots/unfolding/bias_test/Bias_normalised_xsection_electron_abs_lepton_eta_coarse_at_13TeV.pdf} \\
	\includegraphics[width=0.32\textwidth]{/Users/db0268/Mount/SoolinScratch/DPS/DPSTestingGround/DailyPythonScripts/plots/unfolding/closure_test/TUnfold/number_of_unfolded_events_electron_closure_test_for_NJets_at_13TeV.pdf}
	\includegraphics[width=0.32\textwidth]{/Users/db0268/Mount/SoolinScratch/DPS/DPSTestingGround/DailyPythonScripts/plots/unfolding/bias_test/Bias_normalised_xsection_electron_NJets_at_13TeV.pdf} \\
	\caption[help]{help}
	\label{fig:ClosureBiase2}
\end{figure*}

\begin{figure*}[htpb]
	\centering
	\includegraphics[width=0.32\textwidth]{/Users/db0268/Mount/SoolinScratch/DPS/DPSTestingGround/DailyPythonScripts/plots/unfolding/closure_test/TUnfold/number_of_unfolded_events_muon_closure_test_for_HT_at_13TeV.pdf}
	\includegraphics[width=0.32\textwidth]{/Users/db0268/Mount/SoolinScratch/DPS/DPSTestingGround/DailyPythonScripts/plots/unfolding/bias_test/Bias_normalised_xsection_muon_HT_at_13TeV.pdf} \\
	\includegraphics[width=0.32\textwidth]{/Users/db0268/Mount/SoolinScratch/DPS/DPSTestingGround/DailyPythonScripts/plots/unfolding/closure_test/TUnfold/number_of_unfolded_events_muon_closure_test_for_ST_at_13TeV.pdf}
	\includegraphics[width=0.32\textwidth]{/Users/db0268/Mount/SoolinScratch/DPS/DPSTestingGround/DailyPythonScripts/plots/unfolding/bias_test/Bias_normalised_xsection_muon_ST_at_13TeV.pdf} \\
	\includegraphics[width=0.32\textwidth]{/Users/db0268/Mount/SoolinScratch/DPS/DPSTestingGround/DailyPythonScripts/plots/unfolding/closure_test/TUnfold/number_of_unfolded_events_muon_closure_test_for_MET_at_13TeV.pdf}
	\includegraphics[width=0.32\textwidth]{/Users/db0268/Mount/SoolinScratch/DPS/DPSTestingGround/DailyPythonScripts/plots/unfolding/bias_test/Bias_normalised_xsection_muon_MET_at_13TeV.pdf} \\
	\includegraphics[width=0.32\textwidth]{/Users/db0268/Mount/SoolinScratch/DPS/DPSTestingGround/DailyPythonScripts/plots/unfolding/closure_test/TUnfold/number_of_unfolded_events_muon_closure_test_for_WPT_at_13TeV.pdf}
	\includegraphics[width=0.32\textwidth]{/Users/db0268/Mount/SoolinScratch/DPS/DPSTestingGround/DailyPythonScripts/plots/unfolding/bias_test/Bias_normalised_xsection_muon_WPT_at_13TeV.pdf} \\
	\caption[help]{help}
	\label{fig:ClosureBiasmu1}
\end{figure*}

\begin{figure*}[htpb]
	\centering
	\includegraphics[width=0.32\textwidth]{/Users/db0268/Mount/SoolinScratch/DPS/DPSTestingGround/DailyPythonScripts/plots/unfolding/closure_test/TUnfold/number_of_unfolded_events_muon_closure_test_for_lepton_pt_at_13TeV.pdf}
	\includegraphics[width=0.32\textwidth]{/Users/db0268/Mount/SoolinScratch/DPS/DPSTestingGround/DailyPythonScripts/plots/unfolding/bias_test/Bias_normalised_xsection_muon_lepton_pt_at_13TeV.pdf} \\
	\includegraphics[width=0.32\textwidth]{/Users/db0268/Mount/SoolinScratch/DPS/DPSTestingGround/DailyPythonScripts/plots/unfolding/closure_test/TUnfold/number_of_unfolded_events_muon_closure_test_for_abs_lepton_eta_coarse_at_13TeV.pdf}
	\includegraphics[width=0.32\textwidth]{/Users/db0268/Mount/SoolinScratch/DPS/DPSTestingGround/DailyPythonScripts/plots/unfolding/bias_test/Bias_normalised_xsection_muon_abs_lepton_eta_coarse_at_13TeV.pdf} \\
	\includegraphics[width=0.32\textwidth]{/Users/db0268/Mount/SoolinScratch/DPS/DPSTestingGround/DailyPythonScripts/plots/unfolding/closure_test/TUnfold/number_of_unfolded_events_muon_closure_test_for_NJets_at_13TeV.pdf}
	\includegraphics[width=0.32\textwidth]{/Users/db0268/Mount/SoolinScratch/DPS/DPSTestingGround/DailyPythonScripts/plots/unfolding/bias_test/Bias_normalised_xsection_muon_NJets_at_13TeV.pdf} \\
	\caption[help]{help}
	\label{fig:ClosureBiasmu2}
\end{figure*}

\subsubsection{Bottom Line}
\label{sssec:bline}