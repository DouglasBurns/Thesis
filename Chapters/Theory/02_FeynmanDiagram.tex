\section{Representing particles and their interactions}
\label{sec:FD}

The dynamics of particles and their interactions with other particles through the excitations of relativistic quantum fields are not trivial.
To give simple visualisations of these complex interaction processes, Richard Feynman developed a powerful diagrammatic tool in 1948 to directly represent the physics beneath.
This diagrammatic tool is called the Feynman diagram.
A complete Feynman diagram showing a complex example of top quark-antiquark (\ttbar{}) production and decay is shown in Fig.\,\ref{fig:feyn-eg}.
In all Feynman diagrams, time increases from the left-hand side to the right, such that all the initial state particles are represented on the left and final state particles towards the right.
Each arrowed line in the Feynman diagram represents a free fermionic particle of the standard model, each sinusoidal line represents an electroweak mediator particle (\Wboson{}, \Zboson{}, \photon{}) and each coiled line a gluon.
Anti-fermions are represented as fermions moving backwards in time.
The equations of motion of free fermions and anti-fermions are described by the Dirac equation, which is obtained by substituting the Lagrangian density for a Dirac field into the Euler-Lagrange equation. 
% See Appendix/,TODO.
The Lagrangian density for free fermions is given by:
\begin{equation}
\Lagr_{\text{free}}(x) = \overline{\Psi}(x)(i\gamma^{\mu}\partial_{\mu}-m)\Psi(x),
\end{equation}
where $\Psi(x)$ is the fermionic field depending on space-time, $\gamma^{\mu}$ are the gamma matrices, $\partial_{\mu}$ is the partial derivative and $m$ the fermion mass.
The $\Lagr_{\text{free}}$ is invariant under the global phase transformation,
\begin{equation}
\Psi(x) \to e^{iq\chi}\Psi(x),
\end{equation}
leading to the global $\mathrm{U(1)}$ symmetry and by Noether's theorem, the conservation of electric charge.
When a local phase transformation is applied however,
\begin{equation}
\Psi(x) \to e^{iq\chi(x)}\Psi(x),
\end{equation}
\begin{landscape}
\begin{figure*}
\centering
\begin{resizedtikzpicture}{0.9\linewidth}
\begin{feynman}
	\vertex (i1) {\(d\)};
	\vertex [right=3cm of i1](a);
	\vertex [above right=0.75cm and 1.5cm of a] (b){ISR};
	\vertex [below right=0.75cm and 1.5cm of a] (c);

	\vertex [above=2em of i1](i1a){\(u\)};
	\vertex [above=1em of i1](i1b){\(u\)};
	\vertex [right=3cm of i1a](i1ai);
	\vertex [right=3cm of i1b](i1bi);
	\vertex [above right=0.75cm and 1.5cm of i1ai] (i1aii);
	\vertex [above right=0.75cm and 1.5cm of i1bi] (i1bii);

	\vertex [right=1cm of i1a](g1);
	\vertex [right=2cm of i1a](g2);

	\vertex [below=5em of i1](i2){\(u\)};
	\vertex [right=1.5cm of i2] (d);
	\vertex [above right=0.75cm and 1.5cm of d] (e);
	\vertex [below right=0.75cm and 1.5cm of d] (f);
	\vertex [below right=0.75cm and 1.5cm of e] (g);
	\vertex [above right=0.75cm and 1.5cm of e] (h);

	\vertex [below=1em of i2](i2a){\(u\)};
	\vertex [below=2em of i2](i2b){\(d\)};
	\vertex [right=1.5cm of i2a](i2ai);
	\vertex [right=1.5cm of i2b](i2bi);
	\vertex [below right=0.75cm and 1.5cm of i2ai] (i2aii);
	\vertex [below right=0.75cm and 1.5cm of i2bi] (i2bii);

	\vertex [right=2cm of h] (i);
	\vertex [above right=1.5cm and 1.5cm of i] (j);
	\vertex [below right=1.5cm and 1.5cm of i] (k);
	\vertex [above right=1.2cm and 1.5cm of j] (j1);
	\vertex [below right=1.2cm and 1.5cm of j] (j2);
	\vertex [above right=1.2cm and 1.5cm of k] (k1);
	\vertex [below right=1.2cm and 1.5cm of k] (k2);

	\vertex [above right=0.5cm and 1.5cm of j1] (l1){\(\ell\)};
	\vertex [below right=0.5cm and 1.5cm of j1] (l2){\(\overline \nu_\ell\)};
	\vertex [above right=0.5cm and 1.5cm of k1] (m1){\(q\)};
	\vertex [below right=0.5cm and 1.5cm of k1] (m2){\(\overline q'\)};
	\vertex [above right=0.5cm and 1.5cm of k2] (o1){\(\overline b\)};
	\vertex [below right=0.5cm and 1.5cm of k2] (o2){FSR};

	\diagram* {
	(i1) -- [fermion] (a) -- [fermion, edge label=\(q\)] (h),
	(g1) -- [gluon, half left] (g2),
	(a) -- [gluon] (b),
	(i2) -- [fermion] (d) -- [gluon] (e),
	(d) -- [fermion] (f),
	(e) -- [fermion] (g),
	(h) -- [fermion, edge label=\(\overline q\)] (e),
	(h) -- [gluon] (i),
	(i) -- [fermion, edge label=\(t\)] (j),
	(k) -- [fermion, edge label=\(\overline t\)] (i),
	(j) -- [boson, edge label=\(W^+\)] (j1),
	(j) -- [fermion, edge label=\(b\)] (j2),
	(k) -- [boson, edge label=\(W^-\)] (k1),
	(k2) -- [fermion, edge label=\(\overline b\)] (k),
	(j1) -- [fermion] (l1),
	(l2) -- [fermion] (j1),
	(k1) -- [fermion] (m1),
	(m2) -- [fermion] (k1),
	(o1) -- [fermion] (k2),
	(k2) -- [gluon] (o2),
	(i1a) -- [fermion] (i1ai) -- [fermion] (i1aii),
	(i1b) -- [fermion] (i1bi) -- [fermion] (i1bii),
	(i2a) -- [fermion] (i2ai) -- [fermion] (i2aii),
	(i2b) -- [fermion] (i2bi) -- [fermion] (i2bii),
	};
\end{feynman}
\end{resizedtikzpicture}
\caption[A Feynman diagram showing the \ttbar{} pair production by quark-anti-quark annihilation and decay into the single lepton final state (See Section\,\ref{sub:top_quark_decay}). Also shown are soft radiative processes in both the initial (ISR) and final state (FSR). Within the proton, there are also many \QCD{} interactions occuring, such that the colliding partons in the initial state are governed by the parton distribution function (See Section\,\ref{sub:top_quark_production}). All Feynman diagrams have been created using the feyntikz package.]{A Feynman diagram showing the \ttbar{} pair production by quark-anti-quark annihilation and decay into the single lepton final state (See Section\,\ref{sub:top_quark_decay}). Also shown are soft radiative processes in both the initial (ISR) and final state (FSR). Within the proton, there are also many \QCD{} interactions occuring, such that the colliding partons in the initial state are governed by the parton distribution function (See Section\,\ref{sub:top_quark_production}). All Feynman diagrams have been created using the feyntikz package \cite{feyntikz}. }
\label{fig:feyn-eg}
\end{figure*}
\end{landscape}

where the phase $q\chi(x)$ depends on space-time, the Lagrangian is no longer invariant, but contains the additional term:
\begin{equation}
\Lagr_{\text{free}}(x) = \Lagr_{\text{free}}(x)-q\overline{\Psi}(x)\gamma^{\mu}\partial_{\mu}(\chi(x))\Psi(x).
\end{equation}
This violation is solved by the introduction of a new gauge vector field, $A_{\mu}$, and coupling constant $q$ into the Lagrangian density, via the \textit{covariant derivative}
\begin{equation}
D_{\mu} = \partial_{\mu}+iqA_{\mu}.
\end{equation}
The new gauge field interacts with the fermion field $\Psi$, via a single gauge boson (the photon), cancelling the additional term and maintaining gauge invariance, provided the photon is massless.
The combination of the free fermionic fields, photon fields and the EM interaction term is known as \textit{Quantum Electrodynamics} (\QED{}), with the full Lagrangian density given as:
\begin{equation}
\Lagr_{\mathrm{QED}}=\overline{\Psi}(i\gamma^{\mu}\partial_{\mu}-m)\Psi - \frac{1}{4}F^{\mu \nu }F_{\mu \nu } + q\overline{\Psi}\gamma^{\mu}\Psi A_{\mu},
\end{equation}
where $F^{\mu \nu}$ is the EM field strength tensor with the term describing free photons.
% ASIDE SEE APP
% U(1) The generator is a 1x1 matrix (theta)

