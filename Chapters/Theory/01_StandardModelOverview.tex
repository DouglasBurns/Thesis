\section{The components of the standard model}
\label{sec:SM}

The SM contains twelve fermions, split into leptons (\lepton{}) and quarks (\quark{}),  whose interactions are mediated by four gauge bosons as well as the famous Brout-Englert-Higgs (\BEH) boson (\Hboson). 
The set of leptons exist in three generations: electrons, muons and tau particles (\electron{}, \muon{}, \tauon{}), together with their associated neutrinos, (\nue{}, \numu{}, \nutau{}). 
The set of quarks (\quark{}) also exist in three generations: up and down (\uquark{}, \dquark{}), strange and charm (\squark{}, \cquark{}) and the bottom and top (\bquark{}, \tquark{}) particles. 
For each fermion (\fermion{}) that exists there is an anti-fermion (\antifermion{}), with identical quantum properties except conjugated charge and spin.
The properties of the fermionic particles are shown in Table\,\ref{tb:SM_Fermions}.
% These particles obey Fermi-Dirac statistics and are hence known as fermions. 
% $\frac{1}{e^{\frac{(\epsilon_{i}\mu)}{kT}}}$
% $\epsilon_{i}$ is the energy of the i'th state, $\mu$ is the total chemical potential
% At T=0, $\mu = E_{F}+\text{PotEnergyPerElec}$
% In a semiconductor, $\mu$ is the point of symmetry - known as the Fermi Level.

\begin{table}
	\centering
	\footnotesize
	\caption{The fermionic, spin $\sfrac{1}{2}$, particles of the SM and their properties. They are split into three generations of leptons and quarks.}
	\label{tb:SM_Fermions}
	\begin{tabular}{ccccccc}
								& 						& \textbf{Leptons} 			&										&						& \textbf{Quarks} 			& \\
		\textbf{Generation} 	&	\textbf{Particle} 	& \textbf{Mass (\MeVcc{})} 	& \textbf{Charge (\electron{})}			& \textbf{Particle} 	& \textbf{Mass (\MeVcc{})} 	& \textbf{Charge (\electron{})} \\
		\hline
		1 						& \electron{} 			& 0.511  					&	-1									& \uquark{} 			& 2.2						& $\sfrac{2}{3}$ \\
		1 						& \nue{}			 	& $<$2eV  					&	0									& \dquark{} 			& 4.7						& $-\sfrac{1}{3}$ \\
		2 						& \muon{} 		 		& 105.7	 					&	-1									& \squark{}	 			& 96						& $-\sfrac{1}{3}$ \\	
		2 						& \numu{} 				& $<$0.19  					&	0									& \cquark{}	 			& 1280						& $\sfrac{2}{3}$ \\
		3 						& \tauon{}	 	 		& 1777 	 					&	-1									& \bquark{}	 			& 4180						& $-\sfrac{1}{3}$ \\	
		3 						& \nutau{}	 	 		& $<$18.2  					&	0									& \tquark{}	 			& 173100					& $\sfrac{2}{3}$ \\
	\end{tabular}
\end{table}

A particle can interact with another particle through the four fundamental forces: the electromagnetic (EM), weak and strong forces, together with gravity. 
In particle physics, these forces are described by quantum field theories, in which the interactions between fermions are mediated by the exchange of a virtual boson.
% Virtual is a mathematical construct combining time orderings and polarisations etc.
The EM force is mediated by the photon (\photon{}), the weak force by the massive \Wboson{} and \Zboson{} bosons (\Wboson{}, \Zboson{}) and the strong force by a set of eight colourless gluons (\gluon{}). 
Finally, the last particle of the current standard model is the \Hboson{}, which is the mediator between a particle and the \BEH{} field.
The interaction with the \BEH{} field, via the \BEH{} mechanism, is responsible for imparting mass to the fundamental particles. 
The properties of the bosons are shown in Table \ref{tb:SM_Bosons}.
% These force mediating particles obey Bose-Einstein statistics and are hence known as Bosons.

\begin{table}
	\centering
	\footnotesize
	\caption{The bosonic, integer spin, particles of the SM and their properties. The force mediators are spin 1 and the \Hboson{} boson spin 0}
	\label{tb:SM_Bosons}
	\begin{tabular}{cccc}
		\textbf{Force Mediator} 	& \textbf{Particle} 	& \textbf{Mass (\GeVcc{})} 	& \textbf{Charge (\electron{})} \\
		\hline
		EM 							& \photon{} 			& 0 						& 0 \\	 
		Weak 						& \Wboson{} 			& 80.385 					& $\pm1$ \\
									& \Zboson{}	 			& 91.188 					& 0 \\	
		Strong 						& \gluon{}				& 0 						& 0	\\	 
		---							& \Hboson{}				& 125.09 					& 0 \\	 
	\end{tabular}
\end{table}

While the SM is able to describe three of the four fundamental forces between particles, it is not a complete model.
It is unable to reconcile the act of gravity at a fundamental particle level.
This is one of the discrepancies seen between the SM and nature, and must be one of the future evolutions of the SM.
% TODO What theories??? Ouchy question.

