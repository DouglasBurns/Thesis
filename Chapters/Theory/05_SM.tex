
\section{The standard model and its shortcomings} % (fold)
\label{sec:the_full_standard_model}
The full \SM{} combines the \QCD{} and \EWK{} field theories into the combined symmetry group \ensuremath{\mathrm{SU(3)_{c}}\otimes\mathrm{SU(2)_{L}}\otimes\mathrm{U(1)_{Y}}}.
This has the effect of making the full covariant derivative
\begin{equation}
	D_{\mu} = \partial_{\mu}+ig_{S}\frac{\lambda_{a}}{2}G_{a}+ig_{W}\frac{\sigma_{i}}{2}W_{\mu}^{i}+ig'\frac{Y_{W}}{2}B_{\mu},
\end{equation}
and the full Lagrangian density the sum of all individual components
\begin{equation}
	\Lagr_{\SM} = \Lagr_{\QCD} + \Lagr_{\EWK} + \Lagr_{\mathrm{H}} + \Lagr_{\mathrm{Y}}.
\end{equation}
It includes terms for free fermions and gauge bosons and the interactions between them, in addition to mass terms arising from the introduction of the \BEH{} field and a non-zero $v$.
It is the most complete model we have of our Universe, describing fundamental matter and its interactions to an unprecented degree of precision, however as mentioned earlier, it is far from a complete description.
% section the_full_standard_model (end)

\subsection{Gravity} % (fold)
\label{sub:gravity}

Perhaps the largest omission from the \SM{} is the fundamental force of gravity, which all massive particles experience.
In order to include a description of gravity in the \SM{}, a unification with the theory of general relativity at the quantum scale is needed.
The formulation of this unification at present is unknown, however one popular way to do this is with string theory.
Unfortunately, the effect of gravity is completely negligable at the current energy scales being probed and as such it is not experimentally feasible to test any unification theory.  
% $\Lamdba_{P} \sim \ten{19} \GeV$
% subsection gravity (end)

\subsection{Massive neutrinos} % (fold)
\label{sub:massive_neutrinos}

The observation for the interactions with only left-handed neutrinos, results in the Yukawa component of the lagrangian density, shown in Eq.\,\ref{eq:LY}, losing gauge invariance unless the neutrinos are massless.
Experiments such as SNO, which measured the a third of the number of \nue{} expected from production in the Sun, or Super-Kamiokande, which measured half of the number of upward atmospheric \numu{} as downward ones, point to an oscillation in the flavour of the neutrino.
This can be explained by the neutrinos acquiring a mass, such that the flavour states of the neutrinos are then linear superpositions of the mass states.
There is also no reason why right-handed neutrinos should not exist in nature.
% Goldhaber Experiment? https://indico.cern.ch/event/73981/contributions/2080329/attachments/1043729/1487656/Grodzins_Lee_Monday_Opening.pdf 
% subsection massive_neutrinos (end)

\subsection{Dark matter} % (fold)
\label{sub:dark_matter}

Other issues of the \SM{} originate from cosmological observations.
Firstly, the lack of enough mass in galaxies to explain the rotation curves of luminous matter, provides evidence for a type of additional matter which must be massive and electromagnetically inert.
This matter is known as dark matter and is predicted to contribute around 80\% to the matter content of the Universe. 
While the current \SM{} does provide a set of particles which behave in this way, the neutrinos, they are not enough to account for all of the missing matter.

Secondly, the accelerating expansion of the Universe implies that there is a repulsive operation acting.
This process is enshrined under the term dark energy, which accounts for 70\% of the mass-energy content of the Universe.
The \SM{} has no mechanism for describing dark energy.
% subsection dark_matter (end)

\subsection{Baryon asymmetry} % (fold)
\label{sub:baryon_asymmetry}

Another striking problem in the \SM{} is the matter-anti-matter imbalance seen in the Universe.
The Universe is predominantly matter based, however the \SM{} predicts that matter and anti-matter is produced in equal quantities.
Somehow, all the anti-matter present at the beginning of the Universe has been transformed.
While the \SM{} does provide an explanation for some of this transformation, from the charge-parity violating phases in the Cabbibo-Kobayashi-Maskawa (CKM) matrix, it is not enough to account for the entire imbalance.
% Transformed into matter? or dark matter? maybe antimatter couples more stronly to dark matter?
% subsection baryon_asymmetry (end)


\subsection{Heirarchy problem} % (fold)
\label{sub:heirarchy_problem}

The mass of the \Hboson{} boson depends on its bare quark mass and corrections from fermion loops depending on the square of the energy scale.
\begin{equation*}
	m_{\mathrm{H}}^{2} = m_{\mathrm{H, bare}}^{2} + \mathcal{O}(\Lambda^{2})
\end{equation*}
At the \EWK{} scale, these quantum loop corrections are of the same magnitude as the bare mass, however as the energy scale approaches the Planck scale, the corrections diverge.
The very large divergences must be cancelled to keep the observed \Hboson{} boson mass constant.
This requires a significant amount of fine-tuning in the \SM{}.
% 100GeV, 10^19 GeV?
One way this unnatural amount of fine-tuning can be alieviated, is to introduce additional sets of particles which, in part, cancel the loop corrections.
An method of this is supersymmetry.
% subsection heirarchy_problem (end)

% section shortcomings_of_the_standard_model (end)
