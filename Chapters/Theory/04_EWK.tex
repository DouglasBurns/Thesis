\section{The weak interaction and electroweak unification}
\label{sec:EWK}

The weak interaction differs from \QED{} and \QCD{} in that it may violate parity conservation, $\Psi(x)\neq\Psi(-x)$ and change the flavour of a quark. 
% It is also mediated by three massive gauge bosons.
Example weak interactions are highlighted in blue on Fig.\,\ref{fig:feyn-ewk}.
The weak interaction is described by an $\mathrm{SU(2)}$ symmetry group, with the conserved charge being the weak isospin, $T_{3}$.
Only left-handed particles and right-handed anti-particles have been observed to interact via the weak force. 
% Left handed particles are particle s in which the spin directio is opposite the directio of motion
% This leads to the assumption that neutrinos are massless
\begin{figure}[h!]
\centering
\begin{tikzpicture}
\begin{feynman}
	\vertex [black!50](i1) {\(d\)};
	\vertex [right=3cm of i1](a);
	\vertex [black!50,above right=0.75cm and 1.5cm of a] (b){ISR};
	\vertex [below right=0.75cm and 1.5cm of a] (c);

	\vertex [black!50,above=2em of i1](i1a){\(u\)};
	\vertex [black!50,above=1em of i1](i1b){\(u\)};
	\vertex [right=3cm of i1a](i1ai);
	\vertex [right=3cm of i1b](i1bi);
	\vertex [above right=0.75cm and 1.5cm of i1ai] (i1aii);
	\vertex [above right=0.75cm and 1.5cm of i1bi] (i1bii);

	\vertex [right=1cm of i1a](g1);
	\vertex [right=2cm of i1a](g2);

	\vertex [black!50,below=5em of i1](i2){\(u\)};
	\vertex [right=1.5cm of i2] (d);
	\vertex [above right=0.75cm and 1.5cm of d] (e);
	\vertex [below right=0.75cm and 1.5cm of d] (f);
	\vertex [below right=0.75cm and 1.5cm of e] (g);
	\vertex [above right=0.75cm and 1.5cm of e] (h);

	\vertex [black!50,below=1em of i2](i2a){\(u\)};
	\vertex [black!50,below=2em of i2](i2b){\(d\)};
	\vertex [right=1.5cm of i2a](i2ai);
	\vertex [right=1.5cm of i2b](i2bi);
	\vertex [below right=0.75cm and 1.5cm of i2ai] (i2aii);
	\vertex [below right=0.75cm and 1.5cm of i2bi] (i2bii);

	\vertex [right=2cm of h] (i);
	\vertex [above right=1.5cm and 1.5cm of i] (j);
	\vertex [below right=1.5cm and 1.5cm of i] (k);
	\vertex [above right=1.2cm and 1.5cm of j] (j1);
	\vertex [below right=1.2cm and 1.5cm of j] (j2);
	\vertex [above right=1.2cm and 1.5cm of k] (k1);
	\vertex [below right=1.2cm and 1.5cm of k] (k2);

	\vertex [blue!50,above right=0.5cm and 1.5cm of j1] (l1){\(\ell\)};
	\vertex [blue!50,below right=0.5cm and 1.5cm of j1] (l2){\(\overline \nu_\ell\)};
	\vertex [blue!50,above right=0.5cm and 1.5cm of k1] (m1){\(q\)};
	\vertex [blue!50,below right=0.5cm and 1.5cm of k1] (m2){\(\overline q'\)};
	\vertex [black!50,above right=0.5cm and 1.5cm of k2] (o1){\(\overline b\)};
	\vertex [black!50,below right=0.5cm and 1.5cm of k2] (o2){FSR};

	\diagram* {
	(i1) -- [black!50,fermion] (a) -- [black!50,fermion, edge label=\(q\)] (h),
	(g1) -- [black!50,gluon, half left] (g2),
	(a) -- [black!50,gluon] (b),
	(i2) -- [black!50,fermion] (d) -- [black!50,gluon] (e),
	(d) -- [black!50,fermion] (f),
	(e) -- [black!50,fermion] (g),
	(h) -- [black!50,fermion, edge label=\(\overline q\)] (e),
	(h) -- [black!50,gluon] (i),
	(i) -- [blue,very thick,fermion, edge label=\(t\)] (j),
	(k) -- [blue!50,fermion, edge label=\(\overline t\)] (i),
	(j) -- [blue,very thick,boson, edge label=\(W^+\)] (j1),
	(j) -- [blue,very thick,fermion, edge label=\(b\)] (j2),
	(k) -- [blue!50,boson, edge label=\(W^-\)] (k1),
	(k2) -- [blue!50,fermion, edge label=\(\overline b\)] (k),
	(j1) -- [blue!50,fermion] (l1),
	(l2) -- [blue!50,fermion] (j1),
	(k1) -- [blue!50,fermion] (m1),
	(m2) -- [blue!50,fermion] (k1),
	(o1) -- [black!50,fermion] (k2),
	(k2) -- [black!50,gluon] (o2),
	(i1a) -- [black!50,fermion] (i1ai) -- [black!50,fermion] (i1aii),
	(i1b) -- [black!50,fermion] (i1bi) -- [black!50,fermion] (i1bii),
	(i2a) -- [black!50,fermion] (i2ai) -- [black!50,fermion] (i2aii),
	(i2b) -- [black!50,fermion] (i2bi) -- [black!50,fermion] (i2bii),
	};
\end{feynman}
\end{tikzpicture}
\caption[A Feynman diagram showing the \ttbar{} pair production and decay. An example weak interaction is shown in bold blue. Other weak interactions are shown in light blue.]{A Feynman diagram showing the \ttbar{} pair production and decay. An example weak interaction is shown in bold blue. Other weak interactions are shown in light blue.}
\label{fig:feyn-ewk}
\end{figure}
% approx 100GeV
% Y = 2(q-I3)
At higher energies, the electromagnetic and weak interactions unify into the electroweak (\EWK{}) interaction.
This combination results in a \ensuremath{\mathrm{SU(2)_{L}}\otimes\mathrm{U(1)_{Y}}} symmetry group, with three gauge fields, $W^{i}_{\mu}$, introduced by the $\mathrm{SU(2)_{L}}$ generators $\frac{1}{2}\sigma_{i}$ (the Pauli spin matrices) and one gauge field, $B_{\mu}$, from the $\mathrm{U(1)_{Y}}$ generator (weak hypercharge $Y = 2(Q-T_{3})$).
The \EWK{} gauge bosons are formed from the interferences of these fields and are formed as:
\begin{equation}
	W^{\pm}_{\mu} = \frac{1}{\sqrt{2}}(W^{1}_{\mu} \mp iW^{2}_{\mu}),
\end{equation}
\begin{equation}
	Z_{\mu} = W^{3}_{\mu}\cos{\theta_{W}} -  B_{\mu}\sin{\theta_{W}},
\end{equation}
and 
\begin{equation}
	A_{\mu} = W^{3}_{\mu}\cos{\theta_{W}} +  B_{\mu}\sin{\theta_{W}},
\end{equation}
% 4DOUG essentially four fields are created... two charged and two neutral. This fields interfere and the excitations of these inteferences give the gauge bosons.
where $\theta_{W}$ is the weak mixing angle, defined by the ratio of the EM and weak coupling constants 
\begin{equation}
	\tan(\theta_{W}) = \frac{g'}{g_{W}}.	
\end{equation}
The full \EWK{} covariant derivative and Lagrangian density are given by
\begin{equation}
D_{\mu} = \partial_{\mu}+ig_{W}\frac{\sigma_{i}}{2}W_{\mu}^{i}+ig'\frac{Y_{W}}{2}B_{\mu},
\end{equation}
and
\begin{equation}
\Lagr_{\EWK}=\overline{\Psi}i\gamma^{\mu}D_{\mu}\Psi - \frac{1}{4}B^{\mu\nu}B_{\mu\nu} - \frac{1}{4}W^{\mu\nu}_{i}W_{\mu\nu}^{i},
\end{equation}
respectively.
If this is assumed to be an accurate representation of the \EWK{} interaction, it requires that gauge bosons and fermions to be massless in order to preserve the gauge symmetry.
As it is experimentally proven that \Wboson{} and \Zboson{} bosons have mass, a solution is needed.
The problem is solved through a process called \textit{electroweak symmetry breaking}.

\subsection{Electroweak symmetry breaking} % (fold)
\label{sub:electroweak_symmetry_breaking}

Electroweak symmetry breaking solves the mass problem by keeping the particles massless but introducing a new complex scalar field doublet and potential into the Lagrangian density.
% 4 degrees of freedom - three to give mass to weak bosons
% The Higgs mechanism can be summarized by saying that the spontaneous breaking of a gauge theory by a non-zero VEV results in the disappearance of a Goldstone boson and its transformation into the longitudinal component of a massive gauge boson.
% Scalar VEV does not affect electromagnetism
% \begin{equation}
% 	\phi = 	\phi_1+i\phi_2 \\
% \end{equation}
\begin{equation}
	\phi = 
\begin{pmatrix} 
	\phi^{+} \\
	\phi^{0} \\
\end{pmatrix}
	= 
\begin{pmatrix} 
	\phi_1+i\phi_2 \\
	\phi_3+i\phi_4 \\
\end{pmatrix}
\end{equation}
\begin{equation}
\Lagr_{\mathrm{H}}=(D_{\mu}\phi)^{\dagger}(D^{\mu}\phi)-\frac{1}{2}\mu^{2}\phi^{\dagger}\phi - \frac{1}{4}\lambda(\phi^{\dagger}\phi)^{2}
\end{equation}
This potential introduces an infinite set of degenerate non-zero vacuum expectation values $v$, provided $\mu^{2}$ is negative.
\begin{equation}
	v=\left|\sqrt{-\frac{\mu^2}{\lambda}}\right|\approx246\GeV
\end{equation} 
The choice of $v$, conventionally along the positive, real component of $\phi$, \textit{spontaneously} breaks the symmetry of the lagrangian.
By taking this choice, $\phi$ can be rewritten as a perturbation around this point
\begin{equation}
	\phi = \frac{1}{\sqrt{2}}
\begin{pmatrix} 
	0 \\
	v + H(x) \\
\end{pmatrix},
\end{equation}
where $H(x)$ is the massive \BEH{} scalar field, mediated by the \Hboson{} boson with
\begin{equation}
	m_{H} = v\sqrt{2\lambda}.
\end{equation}
The other three degrees of freedom from the intitial scalar field doublet form three massless particles (\textit{goldstone bosons}) which impart mass to the weak bosons, via the \BEH{} mechanism such that:
\begin{equation}
	m_{W} = \frac{vg_{W}}{2},
	\,\,\,
	m_{Z} = \frac{v\sqrt{g_{W}^{2}+g'^{2}}}{2}
\end{equation}

Fermions may also be permitted mass via the Higgs mechanism, by introducing additional terms to the Lagrangian density of the form 
\begin{equation} \label{eq:LY}
	\Lagr_{\mathrm{Y}} = g_{y}(\overline{\Psi}_L\phi\Psi_R+\overline{\Psi}_R\phi^{\dagger}\Psi_L), 
\end{equation}
where $g_{y}$ (\textit{Yukawa coupling}) is the coupling strength of the fermion to the \BEH{} field.
This leads to fermion masses given by
\begin{equation}
	m_{f}=\frac{vg_{y}}{\sqrt{2}}.
\end{equation}
% subsection electroweak_symmetry_breaking (end)
