\chapter{Reweighting the top quark $\mathbf{\mathit{p}_{\mathrm{T}}}$} % (fold)
\label{ch:pt_reweight}

The top quark \pt{} in the \powhegpythia{} model is reweighted up and down to cover the differences between data and simulation in order to provide a basis for the bias tests.
Figures~\ref{fig:Reweighte} and~\ref{fig:Reweightmu} show the reweighted kinematic event distributions in the \eJets{} and \muJets{} channels respectively.

\begin{figure*}[htpb]
	\centering
	\includegraphics[width=0.32\textwidth]{/Users/db0268/Mount/SoolinScratch/DPS/DPSTestingGround/DailyPythonScripts/plots/unfolding/reweighting_check/Reweighting_check_electron_NJets.png} 
	\includegraphics[width=0.32\textwidth]{/Users/db0268/Mount/SoolinScratch/DPS/DPSTestingGround/DailyPythonScripts/plots/unfolding/reweighting_check/Reweighting_check_electron_HT.png} 
	\includegraphics[width=0.32\textwidth]{/Users/db0268/Mount/SoolinScratch/DPS/DPSTestingGround/DailyPythonScripts/plots/unfolding/reweighting_check/Reweighting_check_electron_ST.png} \\
	\includegraphics[width=0.32\textwidth]{/Users/db0268/Mount/SoolinScratch/DPS/DPSTestingGround/DailyPythonScripts/plots/unfolding/reweighting_check/Reweighting_check_electron_MET.png} 
	\includegraphics[width=0.32\textwidth]{/Users/db0268/Mount/SoolinScratch/DPS/DPSTestingGround/DailyPythonScripts/plots/unfolding/reweighting_check/Reweighting_check_electron_WPT.png} \\
	\includegraphics[width=0.32\textwidth]{/Users/db0268/Mount/SoolinScratch/DPS/DPSTestingGround/DailyPythonScripts/plots/unfolding/reweighting_check/Reweighting_check_electron_lepton_pt.png} 
	\includegraphics[width=0.32\textwidth]{/Users/db0268/Mount/SoolinScratch/DPS/DPSTestingGround/DailyPythonScripts/plots/unfolding/reweighting_check/Reweighting_check_electron_abs_lepton_eta_coarse.png} \\
	\caption[The kinematic event distributions given by the \powhegpythia{} sample (green) with the top quark \pt{} reweighted up (red) and down (blue) to cover differences to data (magenta) in the \eJets{} channel. The distributions are normalised to one.]{The kinematic event distributions given by the \powhegpythia{} sample (green) with the top quark \pt{} reweighted up (red) and down (blue) to cover differences to data (magenta) in the \eJets{} channel. The distributions are normalised to one.}
	\label{fig:Reweighte}
\end{figure*}
\begin{figure*}[htpb]
	\centering
	\includegraphics[width=0.32\textwidth]{/Users/db0268/Mount/SoolinScratch/DPS/DPSTestingGround/DailyPythonScripts/plots/unfolding/reweighting_check/Reweighting_check_muon_NJets.png} 
	\includegraphics[width=0.32\textwidth]{/Users/db0268/Mount/SoolinScratch/DPS/DPSTestingGround/DailyPythonScripts/plots/unfolding/reweighting_check/Reweighting_check_muon_HT.png} 
	\includegraphics[width=0.32\textwidth]{/Users/db0268/Mount/SoolinScratch/DPS/DPSTestingGround/DailyPythonScripts/plots/unfolding/reweighting_check/Reweighting_check_muon_ST.png} \\
	\includegraphics[width=0.32\textwidth]{/Users/db0268/Mount/SoolinScratch/DPS/DPSTestingGround/DailyPythonScripts/plots/unfolding/reweighting_check/Reweighting_check_muon_MET.png} 
	\includegraphics[width=0.32\textwidth]{/Users/db0268/Mount/SoolinScratch/DPS/DPSTestingGround/DailyPythonScripts/plots/unfolding/reweighting_check/Reweighting_check_muon_WPT.png} \\
	\includegraphics[width=0.32\textwidth]{/Users/db0268/Mount/SoolinScratch/DPS/DPSTestingGround/DailyPythonScripts/plots/unfolding/reweighting_check/Reweighting_check_muon_lepton_pt.png} 
	\includegraphics[width=0.32\textwidth]{/Users/db0268/Mount/SoolinScratch/DPS/DPSTestingGround/DailyPythonScripts/plots/unfolding/reweighting_check/Reweighting_check_muon_abs_lepton_eta_coarse.png} \\
	\caption[The kinematic event distributions given by the \powhegpythia{} sample (green) with the top quark \pt{} reweighted up (red) and down (blue) to cover differences to data (magenta) in the \muJets{} channel. The distributions are normalised to one.]{The kinematic event distributions given by the \powhegpythia{} sample (green) with the top quark \pt{} reweighted up (red) and down (blue) to cover differences to data (magenta) in the \muJets{} channel. The distributions are normalised to one.}
	\label{fig:Reweightmu}
\end{figure*}


\chapter{Bias in alternate models} % (fold)
\label{ch:bias_in_alternate_models}

Additional bias tests on the unfolding procedure are performed by unfolding \ttbar{} events generated from an alternative model and comparing to the true values.
These are shown in Figs.~\ref{fig:ClosureBiase2} and~\ref{fig:ClosureBiasmu2} in the \eJets{} and \muJets{} channels respectively.
For the most part, they lie comfortably within the systematic uncertainties, shown as the grey band in the ratio plots.
\begin{figure*}[htpb]
	\centering
	\includegraphics[width=0.32\textwidth]{/Users/db0268/Mount/SoolinScratch/DPS/DPSTestingGround/DailyPythonScripts/data_thesis/plots/unfolding/closure_test/normalised_xsection_electron_closure_test_for_NJets_generator.pdf} 
	\includegraphics[width=0.32\textwidth]{/Users/db0268/Mount/SoolinScratch/DPS/DPSTestingGround/DailyPythonScripts/data_thesis/plots/unfolding/closure_test/normalised_xsection_electron_closure_test_for_HT_generator.pdf} 
	\includegraphics[width=0.32\textwidth]{/Users/db0268/Mount/SoolinScratch/DPS/DPSTestingGround/DailyPythonScripts/data_thesis/plots/unfolding/closure_test/normalised_xsection_electron_closure_test_for_ST_generator.pdf} \\
	\includegraphics[width=0.32\textwidth]{/Users/db0268/Mount/SoolinScratch/DPS/DPSTestingGround/DailyPythonScripts/data_thesis/plots/unfolding/closure_test/normalised_xsection_electron_closure_test_for_MET_generator.pdf} 
	\includegraphics[width=0.32\textwidth]{/Users/db0268/Mount/SoolinScratch/DPS/DPSTestingGround/DailyPythonScripts/data_thesis/plots/unfolding/closure_test/normalised_xsection_electron_closure_test_for_WPT_generator.pdf} \\
	\includegraphics[width=0.32\textwidth]{/Users/db0268/Mount/SoolinScratch/DPS/DPSTestingGround/DailyPythonScripts/data_thesis/plots/unfolding/closure_test/normalised_xsection_electron_closure_test_for_lepton_pt_generator.pdf} 
	\includegraphics[width=0.32\textwidth]{/Users/db0268/Mount/SoolinScratch/DPS/DPSTestingGround/DailyPythonScripts/data_thesis/plots/unfolding/closure_test/normalised_xsection_electron_closure_test_for_abs_lepton_eta_coarse_generator.pdf} \\
	\caption[The cross sections for the alternate \ttbar{} production models unfolded using the \powhegpythia{} derived response matrix compared to the true model cross sections are shown for all event variables in the \eJets{} channel in the upper panels. The lower panels give the ratio of the two cross sections known as the bias.]{The cross sections for the alternate \ttbar{} production models unfolded using the \powhegpythia{} derived response matrix compared to the true model cross sections are shown for all event variables in the \eJets{} channel in the upper panels. The lower panels give the ratio of the two cross sections known as the bias.}
	\label{fig:ClosureBiase2}
\end{figure*}
\begin{figure*}[htpb]
	\centering
	\includegraphics[width=0.32\textwidth]{/Users/db0268/Mount/SoolinScratch/DPS/DPSTestingGround/DailyPythonScripts/data_thesis/plots/unfolding/closure_test/normalised_xsection_muon_closure_test_for_NJets_generator.pdf}
	\includegraphics[width=0.32\textwidth]{/Users/db0268/Mount/SoolinScratch/DPS/DPSTestingGround/DailyPythonScripts/data_thesis/plots/unfolding/closure_test/normalised_xsection_muon_closure_test_for_HT_generator.pdf}
	\includegraphics[width=0.32\textwidth]{/Users/db0268/Mount/SoolinScratch/DPS/DPSTestingGround/DailyPythonScripts/data_thesis/plots/unfolding/closure_test/normalised_xsection_muon_closure_test_for_ST_generator.pdf} \\
	\includegraphics[width=0.32\textwidth]{/Users/db0268/Mount/SoolinScratch/DPS/DPSTestingGround/DailyPythonScripts/data_thesis/plots/unfolding/closure_test/normalised_xsection_muon_closure_test_for_MET_generator.pdf} 
	\includegraphics[width=0.32\textwidth]{/Users/db0268/Mount/SoolinScratch/DPS/DPSTestingGround/DailyPythonScripts/data_thesis/plots/unfolding/closure_test/normalised_xsection_muon_closure_test_for_WPT_generator.pdf} \\
	\includegraphics[width=0.32\textwidth]{/Users/db0268/Mount/SoolinScratch/DPS/DPSTestingGround/DailyPythonScripts/data_thesis/plots/unfolding/closure_test/normalised_xsection_muon_closure_test_for_lepton_pt_generator.pdf} 
	\includegraphics[width=0.32\textwidth]{/Users/db0268/Mount/SoolinScratch/DPS/DPSTestingGround/DailyPythonScripts/data_thesis/plots/unfolding/closure_test/normalised_xsection_muon_closure_test_for_abs_lepton_eta_coarse_generator.pdf} \\
	\caption[The cross sections for the alternate \ttbar{} production models unfolded using the \powhegpythia{} derived response matrix compared to the true model cross sections are shown for all event variables in the \muJets{} channel in the upper panels. The lower panels give the ratio of the two cross sections known as the bias.]{The cross sections for the alternate \ttbar{} production models unfolded using the \powhegpythia{} derived response matrix compared to the true model cross sections are shown for all event variables in the \muJets{} channel in the upper panels. The lower panels give the ratio of the two cross sections known as the bias.}
	\label{fig:ClosureBiasmu2}
\end{figure*}

\chapter{Residual distributions}
\label{ch:Res}

Figures~\ref{fig:ResHT},~\ref{fig:ResST},~\ref{fig:ResMET},~\ref{fig:ResWPT},~\ref{fig:ResLPT} and~\ref{fig:ResLETA} show the comparisons of the bin widths and the resolutions (calculated from the residual distributions) in each bin for all event variables except \NJET{}.
The residual distributions are calculated using simulated events from the \powhegpythia{} model where the events from the \eJets{} and \muJets{} channels have been combined.

\begin{figure}[htpb]
	\centering
	\includegraphics[width=0.24\textwidth]{/Users/db0268/Mount/SoolinScratch/DPS/DPSTestingGround/DailyPythonScripts/data_thesis/plots/binning/residuals/combined_HT_0_Residual.pdf}
	\includegraphics[width=0.24\textwidth]{/Users/db0268/Mount/SoolinScratch/DPS/DPSTestingGround/DailyPythonScripts/data_thesis/plots/binning/residuals/combined_HT_1_Residual.pdf}
	\includegraphics[width=0.24\textwidth]{/Users/db0268/Mount/SoolinScratch/DPS/DPSTestingGround/DailyPythonScripts/data_thesis/plots/binning/residuals/combined_HT_2_Residual.pdf} 
	\includegraphics[width=0.24\textwidth]{/Users/db0268/Mount/SoolinScratch/DPS/DPSTestingGround/DailyPythonScripts/data_thesis/plots/binning/residuals/combined_HT_3_Residual.pdf} \\
	\includegraphics[width=0.24\textwidth]{/Users/db0268/Mount/SoolinScratch/DPS/DPSTestingGround/DailyPythonScripts/data_thesis/plots/binning/residuals/combined_HT_4_Residual.pdf} 
	\includegraphics[width=0.24\textwidth]{/Users/db0268/Mount/SoolinScratch/DPS/DPSTestingGround/DailyPythonScripts/data_thesis/plots/binning/residuals/combined_HT_5_Residual.pdf} 
	\includegraphics[width=0.24\textwidth]{/Users/db0268/Mount/SoolinScratch/DPS/DPSTestingGround/DailyPythonScripts/data_thesis/plots/binning/residuals/combined_HT_6_Residual.pdf} \\
	\includegraphics[width=0.24\textwidth]{/Users/db0268/Mount/SoolinScratch/DPS/DPSTestingGround/DailyPythonScripts/data_thesis/plots/binning/residuals/combined_HT_7_Residual.pdf}
	\includegraphics[width=0.24\textwidth]{/Users/db0268/Mount/SoolinScratch/DPS/DPSTestingGround/DailyPythonScripts/data_thesis/plots/binning/residuals/combined_HT_8_Residual.pdf} 
	\includegraphics[width=0.24\textwidth]{/Users/db0268/Mount/SoolinScratch/DPS/DPSTestingGround/DailyPythonScripts/data_thesis/plots/binning/residuals/combined_HT_9_Residual.pdf} \\
	\includegraphics[width=0.24\textwidth]{/Users/db0268/Mount/SoolinScratch/DPS/DPSTestingGround/DailyPythonScripts/data_thesis/plots/binning/residuals/combined_HT_10_Residual.pdf} 
	\includegraphics[width=0.24\textwidth]{/Users/db0268/Mount/SoolinScratch/DPS/DPSTestingGround/DailyPythonScripts/data_thesis/plots/binning/residuals/combined_HT_11_Residual.pdf}
	\includegraphics[width=0.24\textwidth]{/Users/db0268/Mount/SoolinScratch/DPS/DPSTestingGround/DailyPythonScripts/data_thesis/plots/binning/residuals/combined_HT_12_Residual.pdf}
	\caption[The set of residual distributions for the \HT{} event variable calculated using the \powhegpythia{} simulation. Overlaid is the bin width (red) and the resolution (blue)]{The set of residual distributions for the \HT{} event variable calculated using the \powhegpythia{} simulation. Overlaid is the bin width (red) and the resolution (blue)}
	\label{fig:ResHT}
\end{figure}
\begin{figure}[htpb]
	\centering
	\includegraphics[width=0.24\textwidth]{/Users/db0268/Mount/SoolinScratch/DPS/DPSTestingGround/DailyPythonScripts/data_thesis/plots/binning/residuals/combined_ST_0_Residual.pdf}
	\includegraphics[width=0.24\textwidth]{/Users/db0268/Mount/SoolinScratch/DPS/DPSTestingGround/DailyPythonScripts/data_thesis/plots/binning/residuals/combined_ST_1_Residual.pdf}
	\includegraphics[width=0.24\textwidth]{/Users/db0268/Mount/SoolinScratch/DPS/DPSTestingGround/DailyPythonScripts/data_thesis/plots/binning/residuals/combined_ST_2_Residual.pdf} 
	\includegraphics[width=0.24\textwidth]{/Users/db0268/Mount/SoolinScratch/DPS/DPSTestingGround/DailyPythonScripts/data_thesis/plots/binning/residuals/combined_ST_3_Residual.pdf} \\
	\includegraphics[width=0.24\textwidth]{/Users/db0268/Mount/SoolinScratch/DPS/DPSTestingGround/DailyPythonScripts/data_thesis/plots/binning/residuals/combined_ST_4_Residual.pdf} 
	\includegraphics[width=0.24\textwidth]{/Users/db0268/Mount/SoolinScratch/DPS/DPSTestingGround/DailyPythonScripts/data_thesis/plots/binning/residuals/combined_ST_5_Residual.pdf} 
	\includegraphics[width=0.24\textwidth]{/Users/db0268/Mount/SoolinScratch/DPS/DPSTestingGround/DailyPythonScripts/data_thesis/plots/binning/residuals/combined_ST_6_Residual.pdf} \\
	\includegraphics[width=0.24\textwidth]{/Users/db0268/Mount/SoolinScratch/DPS/DPSTestingGround/DailyPythonScripts/data_thesis/plots/binning/residuals/combined_ST_7_Residual.pdf}
	\includegraphics[width=0.24\textwidth]{/Users/db0268/Mount/SoolinScratch/DPS/DPSTestingGround/DailyPythonScripts/data_thesis/plots/binning/residuals/combined_ST_8_Residual.pdf} 
	\includegraphics[width=0.24\textwidth]{/Users/db0268/Mount/SoolinScratch/DPS/DPSTestingGround/DailyPythonScripts/data_thesis/plots/binning/residuals/combined_ST_9_Residual.pdf} \\
	\includegraphics[width=0.24\textwidth]{/Users/db0268/Mount/SoolinScratch/DPS/DPSTestingGround/DailyPythonScripts/data_thesis/plots/binning/residuals/combined_ST_10_Residual.pdf} 
	\includegraphics[width=0.24\textwidth]{/Users/db0268/Mount/SoolinScratch/DPS/DPSTestingGround/DailyPythonScripts/data_thesis/plots/binning/residuals/combined_ST_11_Residual.pdf}
	\includegraphics[width=0.24\textwidth]{/Users/db0268/Mount/SoolinScratch/DPS/DPSTestingGround/DailyPythonScripts/data_thesis/plots/binning/residuals/combined_ST_12_Residual.pdf}
	\caption[The set of residual distributions for the \ST{} event variable calculated using the \powhegpythia{} simulation. Overlaid is the bin width (red) and the resolution (blue)]{The set of residual distributions for the \ST{} event variable calculated using the \powhegpythia{} simulation. Overlaid is the bin width (red) and the resolution (blue)}
	\label{fig:ResST}
\end{figure}
\begin{figure}[htpb]
	\centering
	\includegraphics[width=0.24\textwidth]{/Users/db0268/Mount/SoolinScratch/DPS/DPSTestingGround/DailyPythonScripts/data_thesis/plots/binning/residuals/combined_MET_0_Residual.pdf}
	\includegraphics[width=0.24\textwidth]{/Users/db0268/Mount/SoolinScratch/DPS/DPSTestingGround/DailyPythonScripts/data_thesis/plots/binning/residuals/combined_MET_1_Residual.pdf}
	\includegraphics[width=0.24\textwidth]{/Users/db0268/Mount/SoolinScratch/DPS/DPSTestingGround/DailyPythonScripts/data_thesis/plots/binning/residuals/combined_MET_2_Residual.pdf} \\
	\includegraphics[width=0.24\textwidth]{/Users/db0268/Mount/SoolinScratch/DPS/DPSTestingGround/DailyPythonScripts/data_thesis/plots/binning/residuals/combined_MET_3_Residual.pdf} 
	\includegraphics[width=0.24\textwidth]{/Users/db0268/Mount/SoolinScratch/DPS/DPSTestingGround/DailyPythonScripts/data_thesis/plots/binning/residuals/combined_MET_4_Residual.pdf} 
	\includegraphics[width=0.24\textwidth]{/Users/db0268/Mount/SoolinScratch/DPS/DPSTestingGround/DailyPythonScripts/data_thesis/plots/binning/residuals/combined_MET_5_Residual.pdf} \\
	\caption[The set of residual distributions for the \ptmiss{} event variable calculated using the \powhegpythia{} simulation. Overlaid is the bin width (red) and the resolution (blue)]{The set of residual distributions for the \ptmiss{} event variable calculated using the \powhegpythia{} simulation. Overlaid is the bin width (red) and the resolution (blue)}
	\label{fig:ResMET}
\end{figure}
\begin{figure}[htpb]
	\centering
	\includegraphics[width=0.24\textwidth]{/Users/db0268/Mount/SoolinScratch/DPS/DPSTestingGround/DailyPythonScripts/data_thesis/plots/binning/residuals/combined_WPT_0_Residual.pdf}
	\includegraphics[width=0.24\textwidth]{/Users/db0268/Mount/SoolinScratch/DPS/DPSTestingGround/DailyPythonScripts/data_thesis/plots/binning/residuals/combined_WPT_1_Residual.pdf}
	\includegraphics[width=0.24\textwidth]{/Users/db0268/Mount/SoolinScratch/DPS/DPSTestingGround/DailyPythonScripts/data_thesis/plots/binning/residuals/combined_WPT_2_Residual.pdf} \\
	\includegraphics[width=0.24\textwidth]{/Users/db0268/Mount/SoolinScratch/DPS/DPSTestingGround/DailyPythonScripts/data_thesis/plots/binning/residuals/combined_WPT_3_Residual.pdf} 
	\includegraphics[width=0.24\textwidth]{/Users/db0268/Mount/SoolinScratch/DPS/DPSTestingGround/DailyPythonScripts/data_thesis/plots/binning/residuals/combined_WPT_4_Residual.pdf} \\
	\includegraphics[width=0.24\textwidth]{/Users/db0268/Mount/SoolinScratch/DPS/DPSTestingGround/DailyPythonScripts/data_thesis/plots/binning/residuals/combined_WPT_5_Residual.pdf}
	\includegraphics[width=0.24\textwidth]{/Users/db0268/Mount/SoolinScratch/DPS/DPSTestingGround/DailyPythonScripts/data_thesis/plots/binning/residuals/combined_WPT_6_Residual.pdf} \\
	\caption[The set of residual distributions for the \WPT{} event variable calculated using the \powhegpythia{} simulation. Overlaid is the bin width (red) and the resolution (blue)]{The set of residual distributions for the \WPT{} event variable calculated using the \powhegpythia{} simulation. Overlaid is the bin width (red) and the resolution (blue)}
	\label{fig:ResWPT}
\end{figure}
\begin{figure}[htpb]
	\centering
	\includegraphics[width=0.24\textwidth]{/Users/db0268/Mount/SoolinScratch/DPS/DPSTestingGround/DailyPythonScripts/data_thesis/plots/binning/residuals/combined_lepton_pt_0_Residual.pdf}
	\includegraphics[width=0.24\textwidth]{/Users/db0268/Mount/SoolinScratch/DPS/DPSTestingGround/DailyPythonScripts/data_thesis/plots/binning/residuals/combined_lepton_pt_1_Residual.pdf}
	\includegraphics[width=0.24\textwidth]{/Users/db0268/Mount/SoolinScratch/DPS/DPSTestingGround/DailyPythonScripts/data_thesis/plots/binning/residuals/combined_lepton_pt_2_Residual.pdf} \\
	\includegraphics[width=0.24\textwidth]{/Users/db0268/Mount/SoolinScratch/DPS/DPSTestingGround/DailyPythonScripts/data_thesis/plots/binning/residuals/combined_lepton_pt_3_Residual.pdf} 
	\includegraphics[width=0.24\textwidth]{/Users/db0268/Mount/SoolinScratch/DPS/DPSTestingGround/DailyPythonScripts/data_thesis/plots/binning/residuals/combined_lepton_pt_4_Residual.pdf} 
	\includegraphics[width=0.24\textwidth]{/Users/db0268/Mount/SoolinScratch/DPS/DPSTestingGround/DailyPythonScripts/data_thesis/plots/binning/residuals/combined_lepton_pt_5_Residual.pdf} \\
	\includegraphics[width=0.24\textwidth]{/Users/db0268/Mount/SoolinScratch/DPS/DPSTestingGround/DailyPythonScripts/data_thesis/plots/binning/residuals/combined_lepton_pt_6_Residual.pdf} 
	\includegraphics[width=0.24\textwidth]{/Users/db0268/Mount/SoolinScratch/DPS/DPSTestingGround/DailyPythonScripts/data_thesis/plots/binning/residuals/combined_lepton_pt_7_Residual.pdf}
	\includegraphics[width=0.24\textwidth]{/Users/db0268/Mount/SoolinScratch/DPS/DPSTestingGround/DailyPythonScripts/data_thesis/plots/binning/residuals/combined_lepton_pt_8_Residual.pdf} \\
	\includegraphics[width=0.24\textwidth]{/Users/db0268/Mount/SoolinScratch/DPS/DPSTestingGround/DailyPythonScripts/data_thesis/plots/binning/residuals/combined_lepton_pt_9_Residual.pdf}
	\includegraphics[width=0.24\textwidth]{/Users/db0268/Mount/SoolinScratch/DPS/DPSTestingGround/DailyPythonScripts/data_thesis/plots/binning/residuals/combined_lepton_pt_10_Residual.pdf} 
	\includegraphics[width=0.24\textwidth]{/Users/db0268/Mount/SoolinScratch/DPS/DPSTestingGround/DailyPythonScripts/data_thesis/plots/binning/residuals/combined_lepton_pt_11_Residual.pdf} \\
	\includegraphics[width=0.24\textwidth]{/Users/db0268/Mount/SoolinScratch/DPS/DPSTestingGround/DailyPythonScripts/data_thesis/plots/binning/residuals/combined_lepton_pt_12_Residual.pdf}
	\includegraphics[width=0.24\textwidth]{/Users/db0268/Mount/SoolinScratch/DPS/DPSTestingGround/DailyPythonScripts/data_thesis/plots/binning/residuals/combined_lepton_pt_13_Residual.pdf}
	\includegraphics[width=0.24\textwidth]{/Users/db0268/Mount/SoolinScratch/DPS/DPSTestingGround/DailyPythonScripts/data_thesis/plots/binning/residuals/combined_lepton_pt_14_Residual.pdf} \\
	\includegraphics[width=0.24\textwidth]{/Users/db0268/Mount/SoolinScratch/DPS/DPSTestingGround/DailyPythonScripts/data_thesis/plots/binning/residuals/combined_lepton_pt_15_Residual.pdf}
	\includegraphics[width=0.24\textwidth]{/Users/db0268/Mount/SoolinScratch/DPS/DPSTestingGround/DailyPythonScripts/data_thesis/plots/binning/residuals/combined_lepton_pt_16_Residual.pdf}
	\caption[The set of residual distributions for the \LPT{} event variable calculated using the \powhegpythia{} simulation. Overlaid is the bin width (red) and the resolution (blue)]{The set of residual distributions for the \LPT{} event variable calculated using the \powhegpythia{} simulation. Overlaid is the bin width (red) and the resolution (blue)}
	\label{fig:ResLPT}
\end{figure}
\begin{figure}[htpb]
	\centering
	\includegraphics[width=0.24\textwidth]{/Users/db0268/Mount/SoolinScratch/DPS/DPSTestingGround/DailyPythonScripts/data_thesis/plots/binning/residuals/combined_abs_lepton_eta_0_Residual.pdf}
	\includegraphics[width=0.24\textwidth]{/Users/db0268/Mount/SoolinScratch/DPS/DPSTestingGround/DailyPythonScripts/data_thesis/plots/binning/residuals/combined_abs_lepton_eta_1_Residual.pdf}
	\includegraphics[width=0.24\textwidth]{/Users/db0268/Mount/SoolinScratch/DPS/DPSTestingGround/DailyPythonScripts/data_thesis/plots/binning/residuals/combined_abs_lepton_eta_2_Residual.pdf} \\
	\includegraphics[width=0.24\textwidth]{/Users/db0268/Mount/SoolinScratch/DPS/DPSTestingGround/DailyPythonScripts/data_thesis/plots/binning/residuals/combined_abs_lepton_eta_3_Residual.pdf} 
	\includegraphics[width=0.24\textwidth]{/Users/db0268/Mount/SoolinScratch/DPS/DPSTestingGround/DailyPythonScripts/data_thesis/plots/binning/residuals/combined_abs_lepton_eta_4_Residual.pdf} 
	\includegraphics[width=0.24\textwidth]{/Users/db0268/Mount/SoolinScratch/DPS/DPSTestingGround/DailyPythonScripts/data_thesis/plots/binning/residuals/combined_abs_lepton_eta_5_Residual.pdf} \\
	\includegraphics[width=0.24\textwidth]{/Users/db0268/Mount/SoolinScratch/DPS/DPSTestingGround/DailyPythonScripts/data_thesis/plots/binning/residuals/combined_abs_lepton_eta_6_Residual.pdf}
	\includegraphics[width=0.24\textwidth]{/Users/db0268/Mount/SoolinScratch/DPS/DPSTestingGround/DailyPythonScripts/data_thesis/plots/binning/residuals/combined_abs_lepton_eta_7_Residual.pdf} \\
	\caption[The set of residual distributions for the \LETA{} event variable calculated using the \powhegpythia{} simulation. Overlaid is the bin width (red) and the resolution (blue)]{The set of residual distributions for the \LETA{} event variable calculated using the \powhegpythia{} simulation. Overlaid is the bin width (red) and the resolution (blue)}
	\label{fig:ResLETA}
\end{figure}



\chapter{Extracted $\mathbf{t\overline{t}}$ yields} % (fold)
\label{ch:extracted_ttbar}
Table~\ref{tb:ttyields} shows the extracted \ttbar{} yield per bin for each kinematic event variable.
They are estimated by subtracting of the single top quark, \Vjets{} and data-driven multijet \QCD{} backgrounds from data.
The yields are shown for the \eJets{} and \muJets{} channels together with a combined yield.
The difference in total \ttbar{} yield between variables originates from the differences in the predictions of the \QCD{} background in each variable.
\begin{table}
	\centering
	\caption{The yield of \ttbar{} estimated from the data for each event variable, after the background has been subtracted and fakes removed.}
	\label{tb:ttyields}
	\resizebox{\linewidth}{!}{%
	\begin{tabular}{cc:cc:cc:cc:cc:cc:cc}
		\multicolumn{2}{c:}{\NJET{}}	&	\multicolumn{2}{c:}{\HT}		&	\multicolumn{2}{c:}{\ST}		&	\multicolumn{2}{c:}{\ptmiss}	&	\multicolumn{2}{c:}{\WPT}	&	\multicolumn{2}{c:}{\LPT}	&	\multicolumn{2}{c}{\LETA} 	\\
		\eJets 	& 	\muJets 		&	\eJets 	& 	\muJets 		&	\eJets 	& 	\muJets 		&	\eJets 	& 	\muJets 		&	\eJets 	& 	\muJets 		&	\eJets 	& 	\muJets 		&	\eJets 	& 	\muJets 		\\
		\hline
		\rule{0pt}{1.0em}119093		&	199104		&	24766		&	43648	&	17251	&	34976	&	 93510		&	148677		&	36616	&	 73313	&	24136	&	95549	&	46964	&	68369	\\
		62708		&	102831		&	43925		&	76303	&	46290	&	82681	&	90590		&	154097		&	81633	&	141297	&	56492	&	86796	&	 47170	&	71588	\\
		24527		&	39321		&	46982		&	 78608	&	50965	&	85795	&	 26419		&	45211		&	60153	&	91306	&	44433	&	61162	&	42412	&	62465	\\
		8032		&	12904		&	36136		&	59113	&	38548	&	 60701	&	 5383		&	8450		&	27720	&	38543	&	31077	&	40540	&	 35970	&	 52702	\\
		2392		&	3668		&	24622		&	38966	&	26213	&	 39828	&	1292		&	 2050		&	8535	&	11006	&	 20509	&	26049	&	 22080	&	43279	\\
		975			&	1392		&	16473		&	 25672	&	15801	&	23594	&	588			&	825			&	2280	&	2818	&	13232	&	 16857	&	12358	&	29249	\\
		\NA			&	\NA			&	9941		&	15164	&	9751	&	 13932	&		\NA		&		\NA		&	870	&	1040	&	9155	&	10745	&	7988	&	14065	\\
		\NA			&	\NA			&	6159		&	 9113	&	5671	&	7876	&		\NA		&		\NA		&		\NA		&		\NA		&	5938	&	6966	&	3009	&	17546	\\
		\NA			&	\NA			&	3700		&	5401	&	3115	&	4335	&		\NA		&		\NA		&		\NA		&		\NA		&	3889	&	4565	&	\NA			&	\NA			\\
		\NA			&	\NA			&	2151		&	3100	&	1863	&	2453	&		\NA		&		\NA		&		\NA		&		\NA		&	2653	&	3013	&	\NA			&	\NA			\\
		\NA			&	\NA			&	1107		&	1670	&	998		&	1327	&		\NA		&		\NA		&		\NA		&		\NA		&	1899	&	2100	&	\NA			&	\NA			\\
		\NA			&	\NA			&	715			&	887		&	589		&	742		&		\NA		&		\NA		&		\NA		&		\NA		&	1228	&	1398	&	\NA			&	\NA			\\
		\NA			&	\NA			&	861			&	1222	&	 565	&	755		&		\NA		&		\NA		&		\NA		&		\NA		&	902	&	988	&	\NA			&	\NA			\\
		\NA			&	\NA			&	\NA			&	\NA	&	 \NA	&	\NA		&		\NA		&		\NA		&		\NA		&		\NA		&	585	&	696	&	\NA			&	\NA			\\
		\NA			&	\NA			&	\NA			&	\NA	&	 \NA	&	\NA		&		\NA		&		\NA		&		\NA		&		\NA		&	595	&	614	&	\NA			&	\NA			\\
		\NA			&	\NA			&	\NA			&	\NA	&	 \NA	&	\NA		&		\NA		&		\NA		&		\NA		&		\NA		&	494	&	559	&	\NA			&	\NA			\\
		\NA			&	\NA			&	\NA			&	\NA	&	 \NA	&	\NA		&		\NA		&		\NA		&		\NA		&		\NA		&	538	&	642	&	\NA			&	\NA			\\
	\end{tabular}%
	}
\end{table}

% section extracted_ttbar (end)

\chapter{Largest systematic uncertainties} % (fold)
\label{ch:systApp}

Figures~\ref{fig:Systnorm1} and~\ref{fig:Systnorm2} portray the relative systematic uncertainties in every bin for each variable for the normalised cross section measurements.
Similarly, Figs.~\ref{fig:Systabs1} and~\ref{fig:Systabs2} show the uncertainty compositions for the absolute cross section measurements.
The gold band indicates the total systematic uncertainty and the grey band the total statistical uncertainty.
Shown in bold are the systematic uncertainties which are the leading or subleading contributions to any bin.
\begin{figure*}[hp]
	\centering
	\includegraphics[width=0.85\textwidth]{/Users/db0268/Mount/SoolinScratch/DPS/DPSTestingGround/DailyPythonScripts/data_thesis/plots/systematics/combined/normalised/NJets_systematics_largest.pdf} \\
	\includegraphics[width=0.85\textwidth]{/Users/db0268/Mount/SoolinScratch/DPS/DPSTestingGround/DailyPythonScripts/data_thesis/plots/systematics/combined/normalised/HT_systematics_largest.pdf} \\
	\includegraphics[width=0.85\textwidth]{/Users/db0268/Mount/SoolinScratch/DPS/DPSTestingGround/DailyPythonScripts/data_thesis/plots/systematics/combined/normalised/ST_systematics_largest.pdf} \\
	\caption[The composition of the systematic uncertainties for the \NJET{}, \HT{} and \ST{} event variables. Dominant uncertainties are shown in bold. The grey band represents the total statistical uncertainty and the gold band the total systematic uncertainty.]{The composition of the systematic uncertainties for the \NJET{}, \HT{} and \ST{} event variables. Dominant uncertainties are shown in bold. The grey band represents the total statistical uncertainty and the gold band the total systematic uncertainty.}
	\label{fig:Systnorm1}
\end{figure*}
\begin{figure*}[hp]
	\centering
	\includegraphics[width=0.85\textwidth]{/Users/db0268/Mount/SoolinScratch/DPS/DPSTestingGround/DailyPythonScripts/data_thesis/plots/systematics/combined/normalised/MET_systematics_largest.pdf} \\
	\includegraphics[width=0.85\textwidth]{/Users/db0268/Mount/SoolinScratch/DPS/DPSTestingGround/DailyPythonScripts/data_thesis/plots/systematics/combined/normalised/WPT_systematics_largest.pdf} \\
	\includegraphics[width=0.85\textwidth]{/Users/db0268/Mount/SoolinScratch/DPS/DPSTestingGround/DailyPythonScripts/data_thesis/plots/systematics/combined/normalised/lepton_pt_systematics_largest.pdf} \\
	\includegraphics[width=0.85\textwidth]{/Users/db0268/Mount/SoolinScratch/DPS/DPSTestingGround/DailyPythonScripts/data_thesis/plots/systematics/combined/normalised/abs_lepton_eta_coarse_systematics_largest.pdf} \\
	\caption[The composition of the systematic uncertainties for the \ptmiss{}, \WPT{}, \LPT{} and \LETA{} event variables. Dominant uncertainties are shown in bold. The grey band represents the total statistical uncertainty and the gold band the total systematic uncertainty.]{The composition of the systematic uncertainties for the \ptmiss{}, \WPT{}, \LPT{} and \LETA{} event variables. Dominant uncertainties are shown in bold. The grey band represents the total statistical uncertainty and the gold band the total systematic uncertainty.}
	\label{fig:Systnorm2}
\end{figure*}

\begin{figure*}[hp]
	\centering
	\includegraphics[width=0.85\textwidth]{/Users/db0268/Mount/SoolinScratch/DPS/DPSTestingGround/DailyPythonScripts/data_thesis/plots/systematics/combined/absolute/NJets_systematics_largest.pdf} \\
	\includegraphics[width=0.85\textwidth]{/Users/db0268/Mount/SoolinScratch/DPS/DPSTestingGround/DailyPythonScripts/data_thesis/plots/systematics/combined/absolute/HT_systematics_largest.pdf} \\
	\includegraphics[width=0.85\textwidth]{/Users/db0268/Mount/SoolinScratch/DPS/DPSTestingGround/DailyPythonScripts/data_thesis/plots/systematics/combined/absolute/ST_systematics_largest.pdf} \\
	\caption[The composition of the systematic uncertainties for the \NJET{}, \HT{} and \ST{} event variables. Dominant uncertainties are shown in bold. The grey band represents the total statistical uncertainty and the gold band the total systematic uncertainty.]{The composition of the systematic uncertainties for the \NJET{}, \HT{} and \ST{} event variables. Dominant uncertainties are shown in bold. The grey band represents the total statistical uncertainty and the gold band the total systematic uncertainty.}
	\label{fig:Systabs1}
\end{figure*}
\begin{figure*}[hp]
	\centering
	\includegraphics[width=0.85\textwidth]{/Users/db0268/Mount/SoolinScratch/DPS/DPSTestingGround/DailyPythonScripts/data_thesis/plots/systematics/combined/absolute/MET_systematics_largest.pdf} \\
	\includegraphics[width=0.85\textwidth]{/Users/db0268/Mount/SoolinScratch/DPS/DPSTestingGround/DailyPythonScripts/data_thesis/plots/systematics/combined/absolute/WPT_systematics_largest.pdf} \\
	\includegraphics[width=0.85\textwidth]{/Users/db0268/Mount/SoolinScratch/DPS/DPSTestingGround/DailyPythonScripts/data_thesis/plots/systematics/combined/absolute/lepton_pt_systematics_largest.pdf} \\
	\includegraphics[width=0.85\textwidth]{/Users/db0268/Mount/SoolinScratch/DPS/DPSTestingGround/DailyPythonScripts/data_thesis/plots/systematics/combined/absolute/abs_lepton_eta_coarse_systematics_largest.pdf} \\
	\caption[The composition of the systematic uncertainties for the \ptmiss{}, \WPT{}, \LPT{} and \LETA{} event variables. Dominant uncertainties are shown in bold. The grey band represents the total statistical uncertainty and the gold band the total systematic uncertainty.]{The composition of the systematic uncertainties for the \ptmiss{}, \WPT{}, \LPT{} and \LETA{} event variables. Dominant uncertainties are shown in bold. The grey band represents the total statistical uncertainty and the gold band the total systematic uncertainty.}
	\label{fig:Systabs2}
\end{figure*} 


\chapter{Additional double differential distributions} % (fold)
\label{ch:DD}
Figures~\ref{fig:ST_MET} and~\ref{fig:MET_NJET} show the basic double differential cross section measurements for two additional combinations beyond (\ST{}, \NJET{}) and (\LPT{}, \LETA{}).
These are (\ST{}, \MET{}) and (\MET{}, \NJET{}) respectively.
The first gives a comparison for how well the total activity of a \ttbar{} event is modelled depending on the missing \pt{} and the second how well the missing \pt{} is modelled for different jet multiplicities.

\begin{figure}[htpb]
	\centering
	\includegraphics[width=0.825\textwidth]{/Users/db0268/Mount/SoolinScratch/DPS/DPSTestingGround/DailyPythonScripts/DoubleDiffs/ST_MET/plots/Normalisation_electron.png} \\
	\includegraphics[width=0.825\textwidth]{/Users/db0268/Mount/SoolinScratch/DPS/DPSTestingGround/DailyPythonScripts/DoubleDiffs/ST_MET/plots/Normalisation_muon.png} \\
	\vspace{0.8cm}
	\includegraphics[width=0.4\textwidth]{/Users/db0268/Mount/SoolinScratch/DPS/DPSTestingGround/DailyPythonScripts/DoubleDiffs/ST_MET/unfolding/electron_Response.pdf}
	\includegraphics[width=0.4\textwidth]{/Users/db0268/Mount/SoolinScratch/DPS/DPSTestingGround/DailyPythonScripts/DoubleDiffs/ST_MET/unfolding/muon_Response.pdf} \\
	\vspace{0.8cm}
	\includegraphics[width=0.825\textwidth]{/Users/db0268/Mount/SoolinScratch/DPS/DPSTestingGround/DailyPythonScripts/DoubleDiffs/ST_MET/plots/normalised_XSection_combined.png} \\
	\includegraphics[width=0.825\textwidth]{/Users/db0268/Mount/SoolinScratch/DPS/DPSTestingGround/DailyPythonScripts/DoubleDiffs/ST_MET/plots/absolute_XSection_combined.png} \\
	\vspace{0.4cm}
	\caption[Double differential \ttbar{} production cross section measurements with repect to \ST{} and \ptmiss{}. The upper two panels show the distributions after full event selection for the \eJets{} and \muJets{} channels respectively. The central two panels show the migration in terms of the bin number and the lower two panels show the normalised and absolute cross section measurements in comparison to the \powhegpythia{} model.]{Double differential \ttbar{} production cross section measurements with repect to \ST{} and \ptmiss{}. The upper two panels show the distributions after full event selection for the \eJets{} and \muJets{} channels respectively. The central two panels show the migration in terms of the bin number and the lower two panels show the normalised and absolute cross section measurements in comparison to the \powhegpythia{} model.}
	\label{fig:ST_MET}
\end{figure}
\begin{figure}[htpb]
	\centering
	\includegraphics[width=0.825\textwidth]{/Users/db0268/Mount/SoolinScratch/DPS/DPSTestingGround/DailyPythonScripts/DoubleDiffs/MET_NJets/plots/Normalisation_electron.png} \\
	\includegraphics[width=0.825\textwidth]{/Users/db0268/Mount/SoolinScratch/DPS/DPSTestingGround/DailyPythonScripts/DoubleDiffs/MET_NJets/plots/Normalisation_muon.png} \\
	\vspace{0.8cm}
	\includegraphics[width=0.4\textwidth]{/Users/db0268/Mount/SoolinScratch/DPS/DPSTestingGround/DailyPythonScripts/DoubleDiffs/MET_NJets/unfolding/electron_Response.pdf}
	\includegraphics[width=0.4\textwidth]{/Users/db0268/Mount/SoolinScratch/DPS/DPSTestingGround/DailyPythonScripts/DoubleDiffs/MET_NJets/unfolding/muon_Response.pdf} \\
	\vspace{0.8cm}
	\includegraphics[width=0.825\textwidth]{/Users/db0268/Mount/SoolinScratch/DPS/DPSTestingGround/DailyPythonScripts/DoubleDiffs/MET_NJets/plots/normalised_XSection_combined.png} \\
	\includegraphics[width=0.825\textwidth]{/Users/db0268/Mount/SoolinScratch/DPS/DPSTestingGround/DailyPythonScripts/DoubleDiffs/MET_NJets/plots/absolute_XSection_combined.png} \\
	\vspace{0.4cm}
	\caption[Double differential \ttbar{} production cross section measurements with repect to \ptmiss{} and \NJET{}. The upper two panels show the distributions after full event selection for the \eJets{} and \muJets{} channels respectively. The central two panels show the migration in terms of the bin number and the lower two panels show the normalised and absolute cross section measurements in comparison to the \powhegpythia{} model.]{Double differential \ttbar{} production cross section measurements with repect to \ptmiss{} and \NJET{}. The upper two panels show the distributions after full event selection for the \eJets{} and \muJets{} channels respectively. The central two panels show the migration in terms of the bin number and the lower two panels show the normalised and absolute cross section measurements in comparison to the \powhegpythia{} model.}
	\label{fig:MET_NJET}
\end{figure}

% section bias_in_alternate_models (end)

\chapter{EFT studies}
\label{ch:eft}
Figures~\ref{fig:eftNJET},~\ref{fig:eftST},~\ref{fig:eftMET},~\ref{fig:eftWPT},~\ref{fig:eftLPT} and~\ref{fig:eftLETA} show the predictions using a \LO{} EFT theory model simulated with \mgamcLO{} and \pythia{} compared to the absolute differential \ttbar{} cross sections with respect to \NJET{}, \ST{}, \ptmiss{}, \WPT{}, \LPT{} and \LETA{} in the left panels and the corresponding nominal fit for the $\Delta$ \chisq{} as a function of \CTG{} strength in the right panels.
Also shown in the right panels are the best fit value for the \CTG{} strength together with the 68\% and 95\% confidence intervals.
A summary of the 95\% confidence intervals obtained with each event variable is shown in Tab.~\ref{tb:eft}.
\begin{table}[h!]
	\centering
	\caption{The 95\% confidence intervals reached for each event variable}
	\label{tb:eft}
	\begin{tabular}{cc}
		\textbf{Event Variable}		& \textbf{95\% CI} \rule{0pt}{1.0em}\\	
		\hline
		\rule{0pt}{1.0em}\NJET{}	&	$ -1.79 < \CTG < -0.29 $ \\
		\HT{}						&	$ -1.76 < \CTG < 0.28 $ \\
		\ST{}						&	$ -1.47 < \CTG < 0.62 $ \\
		\ptmiss{}					&	$ -1.31 < \CTG < 0.35 $ \\
		\WPT{}						&	$ -1.54 < \CTG < 0.07 $ \\
		\LPT{}						&	$ -1.86 < \CTG < 0.10 $ \\
		\LETA{} 					&	$ -1.27 < \CTG < 0.92 $ \\
	\end{tabular}%
\end{table}

\begin{figure}[htpb]
	\centering
	\includegraphics[width=0.49\textwidth]{/Users/db0268/Mount/SoolinScratch/EFT/CreateSets/RivetPlots/EFTComp/CMS_2018_I1662081/d08-x01-y01.pdf}
	\includegraphics[width=0.435\textwidth]{/Users/db0268/Mount/SoolinScratch/EFT/CreateSets/RivetPlots/eftChi2/NJets_Chi2.pdf}
	\caption[The left panel shows the predictions of EFT simulations using \CTG{} strengths of -2, 0 and +2 for the \NJET{} event variable. The right panel shows nominal fit of the $\Delta \chisq{}$ value between the prediction and the absolute \ttbar{} cross section for differing \CTG{} strengths.]{The left panel shows the predictions of EFT simulations using \CTG{} strengths of -2, 0 and +2 for the \NJET{} event variable. The right panel shows nominal fit of the $\Delta \chisq{}$ value between the prediction and the absolute \ttbar{} cross section for differing \CTG{} strengths.}
	\label{fig:eftNJET}
\end{figure}

\begin{figure}[htpb]
	\centering
	\includegraphics[width=0.49\textwidth]{/Users/db0268/Mount/SoolinScratch/EFT/CreateSets/RivetPlots/EFTComp/CMS_2018_I1662081/d10-x01-y01.pdf}
	\includegraphics[width=0.435\textwidth]{/Users/db0268/Mount/SoolinScratch/EFT/CreateSets/RivetPlots/eftChi2/ST_Chi2.pdf}
	\caption[The left panel shows the predictions of EFT simulations using \CTG{} strengths of -2, 0 and +2 for the \ST{} event variable. The right panel shows nominal fit of the $\Delta \chisq{}$ value between the prediction and the absolute \ttbar{} cross section for differing \CTG{} strengths.]{The left panel shows the predictions of EFT simulations using \CTG{} strengths of -2, 0 and +2 for the \ST{} event variable. The right panel shows nominal fit of the $\Delta \chisq{}$ value between the prediction and the absolute \ttbar{} cross section for differing \CTG{} strengths.}
	\label{fig:eftST}
\end{figure}

\begin{figure}[htpb]
	\centering
	\includegraphics[width=0.49\textwidth]{/Users/db0268/Mount/SoolinScratch/EFT/CreateSets/RivetPlots/EFTComp/CMS_2018_I1662081/d11-x01-y01.pdf}
	\includegraphics[width=0.435\textwidth]{/Users/db0268/Mount/SoolinScratch/EFT/CreateSets/RivetPlots/eftChi2/MET_Chi2.pdf}
	\caption[The left panel shows the predictions of EFT simulations using \CTG{} strengths of -2, 0 and +2 for the \ptmiss{} event variable. The right panel shows nominal fit of the $\Delta \chisq{}$ value between the prediction and the absolute \ttbar{} cross section for differing \CTG{} strengths.]{The left panel shows the predictions of EFT simulations using \CTG{} strengths of -2, 0 and +2 for the \ptmiss{} event variable. The right panel shows nominal fit of the $\Delta \chisq{}$ value between the prediction and the absolute \ttbar{} cross section for differing \CTG{} strengths.}
	\label{fig:eftMET}
\end{figure}

\begin{figure}[htpb]
	\centering
	\includegraphics[width=0.49\textwidth]{/Users/db0268/Mount/SoolinScratch/EFT/CreateSets/RivetPlots/EFTComp/CMS_2018_I1662081/d12-x01-y01.pdf}
	\includegraphics[width=0.435\textwidth]{/Users/db0268/Mount/SoolinScratch/EFT/CreateSets/RivetPlots/eftChi2/WPT_Chi2.pdf}
	\caption[The left panel shows the predictions of EFT simulations using \CTG{} strengths of -2, 0 and +2 for the \WPT{} event variable. The right panel shows nominal fit of the $\Delta \chisq{}$ value between the prediction and the absolute \ttbar{} cross section for differing \CTG{} strengths.]{The left panel shows the predictions of EFT simulations using \CTG{} strengths of -2, 0 and +2 for the \WPT{} event variable. The right panel shows nominal fit of the $\Delta \chisq{}$ value between the prediction and the absolute \ttbar{} cross section for differing \CTG{} strengths.}
	\label{fig:eftWPT}
\end{figure}

\begin{figure}[htpb]
	\centering
	\includegraphics[width=0.49\textwidth]{/Users/db0268/Mount/SoolinScratch/EFT/CreateSets/RivetPlots/EFTComp/CMS_2018_I1662081/d13-x01-y01.pdf}
	\includegraphics[width=0.435\textwidth]{/Users/db0268/Mount/SoolinScratch/EFT/CreateSets/RivetPlots/eftChi2/lepton_pt_Chi2.pdf}
	\caption[The left panel shows the predictions of EFT simulations using \CTG{} strengths of -2, 0 and +2 for the \LPT{} event variable. The right panel shows nominal fit of the $\Delta \chisq{}$ value between the prediction and the absolute \ttbar{} cross section for differing \CTG{} strengths.]{The left panel shows the predictions of EFT simulations using \CTG{} strengths of -2, 0 and +2 for the \LPT{} event variable. The right panel shows nominal fit of the $\Delta \chisq{}$ value between the prediction and the absolute \ttbar{} cross section for differing \CTG{} strengths.}
	\label{fig:eftLPT}
\end{figure}

\begin{figure}[htpb]
	\centering
	\includegraphics[width=0.49\textwidth]{/Users/db0268/Mount/SoolinScratch/EFT/CreateSets/RivetPlots/EFTComp/CMS_2018_I1662081/d14-x01-y01.pdf}
	\includegraphics[width=0.435\textwidth]{/Users/db0268/Mount/SoolinScratch/EFT/CreateSets/RivetPlots/eftChi2/abs_lepton_eta_coarse_Chi2.pdf}
	\caption[The left panel shows the predictions of EFT simulations using \CTG{} strengths of -2, 0 and +2 for the \LETA{} event variable. The right panel shows nominal fit of the $\Delta \chisq{}$ value between the prediction and the absolute \ttbar{} cross section for differing \CTG{} strengths.]{The left panel shows the predictions of EFT simulations using \CTG{} strengths of -2, 0 and +2 for the \LETA{} event variable. The right panel shows nominal fit of the $\Delta \chisq{}$ value between the prediction and the absolute \ttbar{} cross section for differing \CTG{} strengths.}
	\label{fig:eftLETA}
\end{figure}








% \subsection{Local Gauge Transformation}
% Free Dirac Equation:
% \begin{equation*}
% \Lagr_{\text{free}}(x) = \overline{\Psi}(x)(i\gamma^{\mu}\partial_{\mu}-m)\Psi(x),
% \end{equation*}
% Local Gauge Transformation:
% \begin{equation*}
% \Psi(x) \to e^{iq\chi(x)}\Psi(x),
% \end{equation*}
% Apply:
% \begin{equation*}
% \Lagr_{\text{free}} = e^{-iq\chi}\overline{\Psi}(i\gamma^{\mu}\partial_{\mu}-m)e^{iq\chi}\Psi
% \end{equation*}
% Expand:
% \begin{equation*}
% \Lagr_{\text{free}} = i\gamma^{\mu}.e^{-iq\chi}\overline{\Psi}\partial_{\mu}(e^{iq\chi}\Psi)-m.e^{-iq\chi}.e^{iq\chi}\overline{\Psi}\Psi
% \end{equation*}
% \begin{equation*}
% \Lagr_{\text{free}} = i\gamma^{\mu}.e^{-iq\chi}\overline{\Psi}(e^{iq\chi}\partial_{\mu}\Psi+\partial_{\mu}(e^{iq\chi})\Psi)-m\overline{\Psi}\Psi
% \end{equation*}
% \begin{equation*}
% \Lagr_{\text{free}} = i\gamma^{\mu}\overline{\Psi}\partial_{\mu}\Psi+i\gamma^{\mu}.e^{-iq\chi}.\overline{\Psi}.iq\partial_{\mu}\chi.e^{iq\chi}.\Psi)-m\overline{\Psi}\Psi
% \end{equation*}
% Final:
% \begin{equation*}
% \Lagr_{\text{free}} = \Lagr_{\text{free}}-q\gamma^{\mu}.\overline{\Psi}.\partial_{\mu}\chi.\Psi
% \end{equation*}

% \subsection{Local gauge principle} % (fold)
% \label{sub:local_gauge_principle}

% Add in additional field to Dirac equation*:
% \begin{equation*}
% i\gamma^{\mu}(\partial_{\mu}+iqA_{\mu})\Psi(x)-m\Psi(x)=0
% \end{equation*}
% Local Gauge Transformation:
% \begin{equation*}
% \Psi(x) \to e^{iq\chi(x)}\Psi(x)
% \end{equation*}
% Apply:
% \begin{equation*}
% i\gamma^{\mu}(\partial_{\mu}+iqA_{\mu})e^{iq\chi}\Psi-m.e^{iq\chi}\Psi=0
% \end{equation*}
% \begin{equation*}
% i\gamma^{\mu}.\partial_{\mu}(e^{iq\chi}\Psi)+i\gamma^{\mu}.iqA_{\mu}.e^{iq\chi}\Psi-m.e^{iq\chi}\Psi=0
% \end{equation*}
% \begin{equation*}
% i\gamma^{\mu}.e^{iq\chi}.\partial_{\mu}\Psi+i\gamma^{\mu}.iq\partial_{\mu}\chi e^{iq\chi}\Psi-q\gamma^{\mu}.A_{\mu}.e^{iq\chi}\Psi-m.e^{iq\chi}\Psi=0
% \end{equation*}
% \begin{equation*}
% \text{Dirac}_\text{free}-q\gamma^{\mu}\partial_{\mu}\chi\Psi-q\gamma^{\mu}.A_{\mu}\Psi=0
% \end{equation*}
% To maintain invariance the local gauge transformation of $A_{\mu}$ must go as:
% \begin{equation*}
% A_{\mu} \to A_{\mu} - \partial_{\mu}\chi
% \end{equation*}
% So:
% \begin{equation*}
% \text{Dirac}_\text{free}-q\gamma^{\mu}\partial_{\mu}\chi\Psi-q\gamma^{\mu}.A_{\mu}\Psi+q\gamma^{\mu}.\partial_{\mu}\chi\Psi=0
% \end{equation*}
% \begin{equation*}
% =\text{Dirac}_\text{int}
% \end{equation*}
% This new field exhibits the observed gauge invariance of classical electromagentism.
% % subsection local_gauge_principle (end)

% \subsection{Dirac Equation from Lagrangian} % (fold)
% \label{sub:dirac_equation*_from_lagrangian}

% Lagrangian Density for Dirac Field
% \begin{equation*}
% \Lagr = i\overline{\Psi}\gamma^{\mu}\partial_{\mu}\Psi-m\overline{\Psi}\Psi,
% \end{equation*}
% E-L equation
% \begin{equation*}
% \frac{\partial \Lagr}{\partial \Psi} - \frac{\partial}{\partial x^{\mu}} \left[\frac{\partial \Lagr}{\partial(\partial_{\mu}\Psi)}\right] = 0 
% \end{equation*}
% \begin{equation*}
% \frac{\partial \Lagr}{\partial \overline{\Psi}} - \frac{\partial}{\partial x^{\mu}} \left[\frac{\partial \Lagr}{\partial(\partial_{\mu}\overline{\Psi})}\right] = 0
% \end{equation*}
% Sub
% \begin{equation*}
% i\gamma^{\mu}\partial_{\mu}\Psi-m\Psi-\frac{\partial}{\partial x^{\mu}}[0]
% \end{equation*}
% Dirac equation of motion
% \begin{equation*}
% i\gamma^{\mu}\partial_{\mu}\Psi-m\Psi
% \end{equation*}
% % subsection dirac_equation*_from_lagrangian (end)


% \subsection{Free particle solution to Dirac Equation} % (fold)
% \label{sub:free_particle_solution_to_dirac_equation*}
% \begin{equation*}
% 	\Psi_{u} = u(p)e^{i(\mathbf{p.x}-E.t)}\,\,\text{and}\,\,\Psi_{v} = v(p)e^{-i(\mathbf{p.x}-E.t)}
% \end{equation*}
% % subsection free_particle_solution_to_dirac_equation* (end)



% \subsection{Pauli spin matrices} % (fold)
% \label{sub:pauli_spin_matrices}
% \begin{equation*}
% \sigma_{1} = 
% \begin{pmatrix} 
% 	0 	& 	1 \\
% 	1 	& 	0 \\
% \end{pmatrix}
% \sigma_{2} = 
% \begin{pmatrix} 
% 	0 	& 	-i \\
% 	i 	& 	0 \\
% \end{pmatrix}
% \sigma_{3} = 
% \begin{pmatrix} 
% 	1 	& 	0 \\
% 	0 	& 	-1 \\
% \end{pmatrix}
% \end{equation*}
% % subsection pauli_spin_matrices (end)


% \subsection{Gamma matrices and relations} % (fold)
% \label{sub:gamma_matrices_and_relations}

% \begin{equation*}
% \gamma^{0} = 
% \begin{pmatrix} 
% 	1 	& 	0	&	0	&	0 \\
% 	0 	& 	1	&	0	&	0 \\
% 	0 	& 	0	&	-1	&	0 \\
% 	0 	& 	0	&	0	&	-1 \\
% \end{pmatrix}
% \gamma^{1} = 
% \begin{pmatrix} 
% 	0 	& 	0	&	0	&	1 \\
% 	0 	& 	0	&	1	&	0 \\
% 	0 	& 	-1	&	0	&	0 \\
% 	-1 	& 	0	&	0	&	0 \\
% \end{pmatrix}
% \end{equation*}

% \begin{equation*}
% \gamma^{2} = 
% \begin{pmatrix} 
% 	0 	& 	0	&	0	&	-i \\
% 	0 	& 	0	&	i	&	0 \\
% 	0 	& 	i	&	0	&	0 \\
% 	-i 	& 	0	&	0	&	0 \\
% \end{pmatrix}
% \gamma^{3} = 
% \begin{pmatrix} 
% 	0 	& 	0	&	1	&	0 \\
% 	0 	& 	0	&	0	&	-1 \\
% 	-1 	& 	0	&	0	&	0 \\
% 	0 	& 	1	&	0	&	0 \\
% \end{pmatrix}
% \end{equation*}

% \begin{equation*}
% 	(\gamma^{0})^{2} = I
% \end{equation*}
% \begin{equation*}
% 	(\gamma^{k})^{2} = -I
% \end{equation*}
% \begin{equation*}
% 	\gamma^{0\dagger} = \gamma^{0}
% \end{equation*}
% \begin{equation*}
% 	\gamma^{k\dagger} = -\gamma^{k}
% \end{equation*}
% \begin{equation*}
% 	[\gamma^{\mu}, \gamma^{\nu}] = \gamma^{\mu}\gamma^{\nu}+\gamma^{\nu}\gamma^{\mu} = 2g^{\mu\nu}
% \end{equation*}

% \begin{equation*}
% 	\gamma^{5} = i\gamma^{0}\gamma^{1}\gamma^{2}\gamma^{3}
% \begin{pmatrix} 
% 	0 	& 	0	&	1	&	0 \\
% 	0 	& 	0	&	0	&	1 \\
% 	1 	& 	0	&	0	&	0 \\
% 	0 	& 	1	&	0	&	0 \\
% \end{pmatrix}
% \end{equation*}
% \begin{equation*}
% 	(\gamma^{5})^{2} = 1
% \end{equation*}
% \begin{equation*}
% 	\gamma^{5\dagger} = \gamma^{5}
% \end{equation*}
% \begin{equation*}
% 	[\gamma^{5}, \gamma^{\mu}] = -\gamma^{\mu}\gamma^{5}
% \end{equation*}
% % subsection gamma_matrices (end)


% \subsection{Gellman matrices} % (fold)
% \label{sub:gellman_matrices}
% \begin{equation*}
% 	\lambda^{1} = 
% \begin{pmatrix} 
% 	0 	& 	1	&	0 \\
% 	1 	& 	0	&	0 \\
% 	0 	& 	0	&	0 \\
% \end{pmatrix}
% 	\lambda^{2} = 
% \begin{pmatrix} 
% 	0 	& 	-i	&	0 \\
% 	i 	& 	0	&	0 \\
% 	0 	& 	0	&	0 \\
% \end{pmatrix}
% 	\lambda^{3} = 
% \begin{pmatrix} 
% 	1 	& 	0	&	0 \\
% 	0 	& 	-1	&	0 \\
% 	0 	& 	0	&	0 \\
% \end{pmatrix}
% \end{equation*}

% \begin{equation*}
% 	\lambda^{4} = 
% \begin{pmatrix} 
% 	0 	& 	0	&	1 \\
% 	0 	& 	0	&	0 \\
% 	1 	& 	0	&	0 \\
% \end{pmatrix}
% 	\lambda^{5} = 
% \begin{pmatrix} 
% 	0 	& 	0	&	-i \\
% 	0 	& 	0	&	0 \\
% 	i 	& 	0	&	0 \\
% \end{pmatrix}
% 	\lambda^{6} = 
% \begin{pmatrix} 
% 	0 	& 	0	&	0 \\
% 	0 	& 	0	&	1 \\
% 	0 	& 	1	&	0 \\
% \end{pmatrix}
% \end{equation*}

% \begin{equation*}
% 	\lambda^{7} = 
% \begin{pmatrix} 
% 	0 	& 	0	&	0 \\
% 	0 	& 	0	&	-i \\
% 	0 	& 	i	&	0 \\
% \end{pmatrix}
% 	\lambda^{8} = \frac{1}{\sqrt{3}}
% \begin{pmatrix} 
% 	1 	& 	0	&	0 \\
% 	0 	& 	1	&	0 \\
% 	0 	& 	0	&	2 \\
% \end{pmatrix}
% \end{equation*}
% % subsection gellman_matrices (end)


% \subsection{Symmetry breaking} % (fold)
% \label{sub:symmetry_breaking}
% Consider scalar potential:
% \begin{equation*}
% 	V(\phi) = \frac{1}{2}\mu^{2}\phi^{2} + \frac{1}{4}\lambda\phi^{4}
% \end{equation*}
% The corresponding lagrangian is given by:
% \begin{equation*}
% 	\Lagr = T - V
% \end{equation*}
% \begin{equation*}
% 	\Lagr = \frac{1}{2}(\partial_{\mu}\phi)(\partial^{\mu}\phi) - \frac{1}{2}\mu^{2}\phi^{2} - \frac{1}{4}\lambda\phi^{4}
% \end{equation*}
% where $(\partial_{\mu}\phi)(\partial^{\mu}\phi)$ is the kinematic term for a scalar particle, $\mu^{2}\phi^{2}$ represents the mass term and $\lambda\phi^{4}$ term represents the self interactions of the scalar field.

% The vacuum state is the lowest energy state of $\phi$ which corresponds to a minimum of $V(\phi)$. For $\phi$ to generate a minimum $\lambda>0$, however there are no such restrictions on $\mu^{2}$.
% \begin{figure}[htpb]
% 	\centering
% 	\includegraphics[width=0.24\textwidth]{Figures/hpot1}
% 	\includegraphics[width=0.24\textwidth]{Figures/hpot2}
% 	\includegraphics[width=0.24\textwidth]{Figures/hpot3}
% 	\includegraphics[width=0.24\textwidth]{Figures/hpot4}
% \end{figure}
% If $\mu^{2}>0$, this result in a scalar particle of mass $\mu$ with self interaction $\propto\phi^{4}$
% If $\mu^{2}<0$, $\mu$ is no longer interpretated as a mass. The minimum of $\phi$ occurs at degenerate $\pm v = \pm \left|\sqrt{\frac{-\mu^2}{\lambda}}\right|$, which leads to 
% \begin{equation*}
% 	-\lambda v^2 = \mu^2
% \end{equation*}
% The choice of $+v$ or $-v$, breaks the symmetry of the lagrangian. This is spontaneous symmetry breaking.
% Choosing vacuum state of field to be at $+v$, the excitations of the field (decribing particle states) can be obtained through perturbation theory
% \begin{equation*}
% 	\phi(x)\to v + \eta(x), \,\,\,\, \partial_{\mu}\phi = \partial_{\mu}\eta
% \end{equation*}
% The lagrangian becomes
% \begin{equation*}
% 	\Lagr = \frac{1}{2}(\partial_{\mu}\eta)(\partial^{\mu}\eta) - \frac{1}{2}\mu^{2}(v+\eta)^{2} - \frac{1}{4}\lambda(v+\eta)^{4}
% \end{equation*}
% \begin{equation*}
% 	\Lagr = \frac{1}{2}(\partial_{\mu}\eta)(\partial^{\mu}\eta) - \lambda v^2\eta^2 - \lambda v\eta^3 - \frac{1}{4}\lambda \eta^4 + \frac{1}{4}\lambda v^4 
% \end{equation*}
% The term $\propto\eta^2$ is interpretated as a mass. ($\eta$ is a massive scalar field) $m_{\eta} = \sqrt{2\lambda v^2}$.
% Terms term $\propto\eta^3$ and term $\propto\eta^4$ are triple and quartic couplings of the field.
% Thios lagrangian is essentially the same as the initial except with excitations perturbed about a non-zero vacuum expectation value
% % subsection symmetry_breaking (end)

% \subsection{The full standard model} % (fold)
% \label{sub:the_full_standard_model}
% \begin{equation*}
% 	\Lagr_{\SM} = \Lagr_{\QCD} + \Lagr_{\EWK} + \Lagr_{\mathrm{H}} + \Lagr_{\mathrm{Y}}
% \end{equation*}
% \begin{equation*}
% 	\Lagr_{\QCD}=\overline{\Psi}_{i}(i(\gamma^{\mu}D_{\mu})_{ij}-m\delta_{ij})\Psi_{j} - \frac{1}{4}G^{\mu \nu}_{a}G^{a}_{\mu \nu},
% \end{equation*}
% \begin{equation*}
% \Lagr_{\EWK}=\overline{\Psi}i\gamma^{\mu}D_{\mu}\Psi - \frac{1}{4}B^{\mu\nu}B_{\mu\nu} - \frac{1}{4}W^{\mu\nu}_{i}W_{\mu\nu}^{i},
% \end{equation*}
% \begin{equation*}
% 	\Lagr_{\mathrm{H}}=(D_{\mu}\phi)^{\dagger}(D^{\mu}\phi)-\frac{1}{2}\mu^{2}\phi^{\dagger}\phi - \frac{1}{4}\lambda(\phi^{\dagger}\phi)^{2}
% \end{equation*}
% \begin{equation*}
% 	\Lagr_{\mathrm{Y}} = g_{y}\overline{\Psi}_{\mathrm{R}}\phi\Psi_{\mathrm{L}}, 
% \end{equation*}
% % subsection the_full_standard_model (end)

% \subsection{DEBUG:NOETHER} % (fold)
% \label{sub:tmp_noether}
% e.g. m in orbit
% \begin{equation*}
% 	L = \frac{1}{2}mv^{2} + \frac{GMm}{r}
% \end{equation*}
% and in polar coordinates:
% \begin{equation*}
% 	L = \frac{1}{2}m\dot{r}^{2} + \frac{1}{2}mr^{2}\dot{\phi}^{2} + \frac{GMm}{r}
% \end{equation*}
% This equation* of motion is independent of $\phi$ and therefore angularly invariant.
% The Euler-Lagrange Equation is:  
% \begin{equation*}
% 	\frac{d}{dt}\left(\frac{\partial L}{\partial\dot{\phi}}\right) - \frac{\partial L}{d\phi}  = 0
% \end{equation*}
% The conserved current:
% \begin{equation*}
% 	J = \frac{\partial L}{\partial\dot{\phi}} = mr^{2}\dot{\phi}
% \end{equation*}
% \begin{equation*}
% 	\frac{dJ}{dt} = \frac{1}{2}r\ddot{\phi} + \dot{r}\dot{\phi} = 0
% \end{equation*}
% This is equivalent to the conservation of angular momentum

