\begin{figure}[h!]
	\centering
	\begin{subfigure}{.49\textwidth}
		\centering
		\begin{tikzpicture}
		\begin{feynman}
			\vertex [black](i1){\(g\)};
			\vertex [black,right=1.5cm of i1](a);
			\vertex [black,above right=0.75cm and 1.5cm of a](f1){\(t\)};
			\vertex [black,below right=0.75cm and 1.5cm of a](b);
			\vertex [black,right=1.5cm of b](f2);
			\vertex [black,above right=0.75cm and 1.5cm of f2](f2a){\(t\)};
			\vertex [black,below right=0.75cm and 1.5cm of f2](f2b){\(\overline{t}\)};
			\vertex [black,below=1.5cm of i1](i2){\(g\)};
			\vertex [black,right=1.5cm of i2](c);
			\vertex [black,below right=0.75cm and 1.5cm of c] (f3){\(\overline{t}\)};

			\diagram* {
			(i1) -- [black,gluon] (a)
			(a) -- [black,fermion] (f1),
			(a) -- [black,anti fermion] (b),
			(b) -- [black,gluon] (f2),
			(f2) -- [black,fermion] (f2a),
			(f2) -- [black,anti fermion] (f2b),
			(i2) -- [black,gluon] (c),
			(c) -- [black,fermion] (b),
			(c) -- [black,anti fermion] (f3),
			};
		\end{feynman}
		\end{tikzpicture}
	\end{subfigure}
	\begin{subfigure}{.49\textwidth}
		\centering
		\begin{tikzpicture}
		\begin{feynman}
			\vertex [black](i1){\(g\)};
			\vertex [black,right=1.5cm of i1](a);
			\vertex [black,above right=0.75cm and 1.5cm of a](f1){\(\overline{t}\)};
			\vertex [black,below right=0.75cm and 1.5cm of a](c);
			\vertex [black,above right=0.75cm and 1.5cm of b](f2){\(t\)};

			\vertex [black,below=2.5cm of i1](i2){\(g\)};
			\vertex [black,right=1.5cm of i2](b);
			\vertex [black,above right=0.75cm and 1.5cm of b](d);
			\vertex [black,below right=0.75cm and 1.5cm of b](f3){\(t\)};
			\vertex [black,below right=0.75cm and 1.5cm of d](f4){\(\overline{t}\)};

			\diagram* {
			(i1) -- [black,gluon] (a)
			(a) -- [black,anti fermion] (f1),
			(a) -- [black,fermion] (c),
			(c) -- [black,fermion] (f2),
			
			(c) -- [black,gluon] (d),

			(i2) -- [black,gluon] (b)
			(b) -- [black,fermion] (f3),
			(b) -- [black,anti fermion] (d),
			(d) -- [black,anti fermion] (f4),
			};
		\end{feynman}
		\end{tikzpicture}
	\end{subfigure}
	\caption[Feynman diagrams for the production of four top quarks. The mediating boson is predominantly a gluon, however photons, \Zboson{} bosons and \Hboson{} bosons can also mediate.]{Feynman diagrams for the production of four top quarks. The mediating boson is predominantly a gluon, however photons, \Zboson{} bosons and \Hboson{} bosons can also mediate. }
	\label{fig:feyn-tttt}
\end{figure}