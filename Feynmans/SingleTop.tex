\begin{figure}[h!]
	\centering
	\begin{subfigure}{.3\textwidth}
		\centering
		\begin{tikzpicture}
		\begin{feynman}
			\vertex (A);
			\vertex [above left=0.75cm and 1.5cm of A] (ai) {\(\overline{q}'\)};
			\vertex [below left=0.75cm and 1.5cm of A] (aii) {\(q\)};

			\vertex [right=1.5cm of A] (B);
			\vertex [above right=0.75cm and 1.5cm of B] (bi) {\(t\)};
			\vertex [below right=0.75cm and 1.5cm of B] (bii) {\(\overline{b}\)};

			\diagram*{
			(ai) -- [anti fermion] (A),
			(aii) -- [fermion] (A),
			(A) -- [boson, edge label=\(W^{+}\)] (B),
			(B) -- [fermion] (bi),
			(B) -- [anti fermion] (bii),
			};
		\end{feynman}
		\end{tikzpicture}
	\end{subfigure}
	\hspace{0.45cm}
	\begin{subfigure}{.3\textwidth}
		\centering
		\begin{tikzpicture}
		\begin{feynman}
			\vertex [black](i1){\(q\)};
			\vertex [black,right=1.5cm of i1](a);
			\vertex [black,above right=0.75cm and 1.5cm of a](f1){\(q'\)};
			\vertex [black,below right=0.75cm and 1.5cm of a](b);
			\vertex [black,right=1.5cm of b](f2){\(t\)};
			\vertex [black,below=1.5cm of i1](i2){\(g\)};
			\vertex [black,right=1.5cm of i2](c);
			\vertex [black,below right=0.75cm and 1.5cm of c] (f3){\(\overline{b}\)};

			\diagram* {
			(i1) -- [black,fermion] (a) -- [black,fermion] (f1),
			(a) -- [black,boson, edge label=\(W^{+}\)] (b),
			(i2) -- [black,gluon] (c),
			(c) -- [black,fermion] (b),
			(b) -- [black,fermion] (f2),
			(c) -- [black,anti fermion] (f3),
			};
		\end{feynman}
		\end{tikzpicture}
	\end{subfigure}
	\hspace{0.45cm}
	\begin{subfigure}{.3\textwidth}
		\centering
		\begin{tikzpicture}
		\begin{feynman}
			\vertex (A);
			\vertex [above left=0.75cm and 1.5cm of A] (ai) {\(b\)};
			\vertex [below left=0.75cm and 1.5cm of A] (aii) {\(g\)};

			\vertex [right=1.5cm of A] (B);
			\vertex [above right=0.75cm and 1.5cm of B] (bi) {\(W^{-}\)};
			\vertex [below right=0.75cm and 1.5cm of B] (bii) {\(t\)};
			\diagram*{
			(ai) -- [fermion] (A),
			(aii) -- [gluon] (A),
			(A) -- [fermion] (B),
			(B) -- [boson] (bi),
			(B) -- [fermion] (bii),
			};
		\end{feynman}
		\end{tikzpicture}
	\end{subfigure}
	% \hspace{0.45cm}
	% \begin{subfigure}{.32\textwidth}
	% 	\centering
	% 	\begin{tikzpicture}
	% 	\begin{feynman}
	% 		\vertex [black](i1){\(g\)};
	% 		\vertex [black,right=1.5cm of i1](a);
	% 		\vertex [black,above right=0.75cm and 1.5cm of a](f1){\(t\)};
	% 		\vertex [black,below right=0.75cm and 1.5cm of a](b);
	% 		\vertex [black,right=1.5cm of b](f2){\(W^{-}\)};
	% 		\vertex [black,below=1.5cm of i1](i2){\(g\)};
	% 		\vertex [black,right=1.5cm of i2](c);
	% 		\vertex [black,below right=0.75cm and 1.5cm of c] (f3){\(\overline{b}\)};

	% 		\diagram* {
	% 		(i1) -- [black,gluon] (a)
	% 		(a) -- [black,fermion] (f1),
	% 		(a) -- [black,anti fermion] (b),
	% 		(b) -- [black,boson] (f2),
	% 		(i2) -- [black,gluon] (c),
	% 		(c) -- [black,fermion] (b),
	% 		(c) -- [black,anti fermion] (f3),
	% 		};
	% 	\end{feynman}
	% 	\end{tikzpicture}
	% \end{subfigure}
	\caption[Feynman diagrams for the production of single top quarks. The left panel shows single top quark production via the s-channel, the central panel by the t-channel and the right panel in association with a \Wboson{} boson.]{Feynman diagrams for the production of single top quarks. The left panel shows single top quark production via the s-channel, the central panel by the t-channel and the right panel in association with a \Wboson{} boson.}
	\label{fig:feyn-t}
\end{figure}